\section{Тема 15. Поверхность второго порядка в трехмерном пространстве. Задание 3.}
\label{sec:task15}

\begin{enumerate}
	\item\largeПостроить часть параболоида, заданного уравнением $\frac{x^2}{9} + \frac{y^2}{4} = 2z$, лежащую в диапазоне $-3 \leq x \leq 3$, $-2 \leq y \leq 2$ с шагом $\Delta = 0,5$ для обеих переменных.
	\item\largeВыразим \textit{z} из уравнения $\frac{x^2}{9} + \frac{y^2}{4} = 2z$ и получим $z =\frac{x^2}{18} + \frac{y^2}{8}$.
	\begin{item}
		\largeПостроим таблицу данных (x; y; z). Значения \textit{x} заполним в диапазоне от -3 до 3 с шагом 0,5, а значения \textit{y} в диапазоне от -2 до 2 с шагом 0,5. Воспользуемся формулой $z =\frac{x^2}{18} + \frac{y^2}{8}$ и получим все значения \textit{z}.

		\largeВ результате получена таблица, представленная на рисунке 9.
		\includeimage{assets/task_15_table.png}{0.9}{Таблица значений функции $z(x, y)$}

        \newpage

		\largeДля построения части параболоида выберем тип диаграммы -- \textit{Поверхность}, вид -- \textit{Поверхность}. Полученная часть параболоида представлена на рисунке 10.

		\includeimage{assets/task_15_graph.png}{0.9}{Часть параболоида $\frac{x^2}{9} + \frac{y^2}{4} = 2z$}
	\end{item}
\end{enumerate}