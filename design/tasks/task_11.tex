\section{Тема 11. Аналитическая геометрия. Задание 3.}
\label{sec:task11}

\begin{enumerate}
	\item\largeПостроить прямую $3x + 2y - 4 = 0$ в диапазоне $x \in [-1;3]$ с шагом $\Delta = 0,25$

    \begin{item}
		Нахождение уравнения прямой:
		\begin{enumerate}
			\item\largeВыразим \textit{y} из исходного уравнения. Перенесем \textit{y} в правую сторону и получим: $3x - 4 = -2y$
			\item\largeДомножим все уравнение на -1 и получим $-3x + 4 = 2y$
			\item\largeВыразим \textit{y} из полученного уравнения и получим $y = -1.5x + 2$
		\end{enumerate}
	\end{item}

	\begin{item}
		\largeПостроим таблицу данных (x; y). Значения аргумента заполним в диапазоне от -1 до 3 с шагом 0,25. В ячейку $B2$ вводим формулу вида = $-1,5 * A2 + 2$, затем автозаполнением получаем оставшиеся значения функции.

		\largeВ результате получена таблица, представленная на рисунке 1.
		\includeimage{assets/task_11_table.png}{0.22}{Таблица значений функции}

		\largeДля построения прямой выберем тип диаграммы -- \textit{График}, вид -- \textit{График с маркерами}. Полученный график представлен на рисунке 2.

		\includeimage{assets/task_11_graph.png}{0.8}{График прямой $y = -1.5x + 2$ в диапазоне $[-1;3]$}
	\end{item}
\end{enumerate}