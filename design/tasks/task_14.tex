\section{Тема 14. Плоскость в трехмерном пространстве. Задание 3.}
\label{sec:task14}

\begin{enumerate}
	\item\largeПостроить плоскость, проходящую через точки $M_1$(3,3,1), $M_2$(2,3,2), $M_3$(1,1,3), при $-1 \leq x \leq 4$ с шагом $\Delta = 0,5$ и $-1 \leq y \leq 3$ с шагом $\Delta = 1$.
	\item\largeПодставим значения под формулу $i * ((y_2 - y_1) * (z_3 - z_1) - (y_3 - y_1) * (z_2 - z_1)) - j * ((x_2 - x_1) * (z_3 - z_1) - (x_3 - x_1) * (z_2 - z_1)) + k * ((x_2 - x_1) * (y_3 - y_1) - (x_3 - x_1) * (y_2 - y_1))$, где номер переменной означает номер точки, из которой берется значение. Так получаем (2;0;2) -- вектор нормали к плоскости. Подставляем значения под общее уравнение плоскости $Ax + By + Cz + D = 0$ и получаем $2x + 2z + D = 0$. Возьмем значения любой из точек и получим, что $8 + D = 0$, следовательно $D = -8$. Выразим \textit{z} из уравнения $2x + 2z - 8 = 0$ и получим $z = 4 - x$.
	\begin{item}
		\largeПостроим таблицу данных (x; y; z). Значения \textit{x} заполним в диапазоне от -1 до 4 с шагом 0,5, а значения \textit{y} в диапазоне от -1 до 3 с шагом 1. Воспользуемся формулой $z = 4 - x$ и получим все значения \textit{z}.

		\largeВ результате получена таблица, представленная на рисунке 7.
		\includeimage{assets/task_14_table.png}{0.55}{Таблица значений функции $z(x, y)$}

        \newpage

		\largeДля построения плоскости выберем тип диаграммы -- \textit{Поверхность}, вид -- \textit{Проволочная поверхность}. Полученная плоскость представлена на рисунке 8.

		\includeimage{assets/task_14_graph.png}{0.9}{Диаграмма плоскости $2x + 2z - 8 = 0$}
	\end{item}
\end{enumerate}