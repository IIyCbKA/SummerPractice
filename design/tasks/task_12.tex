\section{Тема 12. Кривые второго порядка на плоскости. Задание 3.}
\label{sec:task12}

\begin{enumerate}
	\item\largeПостроить верхнюю часть эллипса $0,1 \leq x \leq 5,1$ с шагом $\Delta = 0,25$, заданного уравнением $\frac{x^2}{4} + y^2 = 1$

    \begin{item}
		Нахождение уравнения кривой:
		\begin{enumerate}
			\item\largeВыразим \textit{y} из исходного уравнения. Перенесем \textit{y} в правую сторону и получим: $1 - \frac{x^2}{4} = y^2$
			\item\largeВыразим \textit{y} из полученного уравнения и получим $y = \sqrt{1 - \frac{x^2}{4}}$
			\item\largeОбратим внимание, что значение выражения $1 - \frac{x^2}{4}$ должно быть $\geq 0$, а следовательно, допустимый диапазон аргумента $x \in [-2;2]$
		\end{enumerate}
	\end{item}

	\begin{item}
		\largeПостроим таблицу данных (x; y). Значения аргумента заполним в диапазоне от 0,1 до 5,1 с шагом 0,25. В ячейку $B2$ вводим формулу вида = $(1 - \frac{(A2) ^ 2}{4}) ^ \frac{1}{2}$, затем автозаполнением получаем оставшиеся значения функции. Значения функции в тех ячейках, при аргументе которых уравнение не имеет решений закрасим красным цветом. Поставим точку с аргументом $x = 2$ как крайнюю, имеющую решение.

		\largeВ результате получена таблица, представленная на рисунке 3.
		\includeimage{assets/task_12_table.png}{0.3}{Таблица значений функции}

        \newpage

		\largeДля построения кривой выберем тип диаграммы -- \textit{График}, вид -- \textit{График с маркерами}. Полученный график представлен на рисунке 4.

		\includeimage{assets/task_12_graph.png}{0.9}{График верхней части эллипса $\frac{x^2}{4} + y^2 = 1$ в диапазоне $[0,1;5,1]$}
	\end{item}
\end{enumerate}