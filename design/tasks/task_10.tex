\section{Тема 10. Алгоритмы обработки символьной информации. Задание 23.}
\label{sec:task10}

\begin{enumerate}
	\item\largeПо правилам пунктуации пробел может стоять после, а не перед каждым из следующих знаков: . , ; : ! ? ) ] \} …; перед, а не после знаков: ( [ \{. Заданный текст проверить на соблюдение этих правил и при необходимости исправить. Вместо пробела может быть перевод строки или знак табуляции.
	\begin{item}
		Словесное описание \textit{алгоритма}:

		\begin{enumerate}
			\item\largeУбираем лишние пробелы после знаков препинания: . , ; : ! ? ) ] \} …;
			\item\largeДобавляем недостающие пробелы после знаков препинания: . , ; : ! ? ) ] \} …;
			\item\largeУбираем пробелы после открывающих скобок: ( [ \{.
			\item\largeУбираем лишние пробелы перед открывающими скобками: ( [ \{.
		\end{enumerate}
	\end{item}
	\begin{item}
		Спецификация функции \textit{CorrectionText}:
		\begin{enumerate}
			\itemЗаголовок: \colorbox{pybg}{\textit{def} CorrectionText(text: str) -> str}
			\itemНазначение: используется для корректировки пробелов перед знаками препинания и после них, согласно правилам пунктуации.
		\end{enumerate}

		Блок-схема:
		\begin{center}
			\begin{tikzpicture}[node distance=1.2cm]
                \node (start) [terminator, minimum width=11.68cm]
			    {\textit{def} CorrectionText(text: str)};

                \node (main)[process, below of=start, yshift=-0.75cm]
				{\begin{tabular}{c}Воспользуемся функцией \textit{sub()} из модуля \textbf{re} и \\ скорректируем знаки препинания с помощью регулярных \\ выражений. Результат запишем в переменную \textbf{text}\end{tabular}};

                \node (return) [data, below of=main, yshift=-1.5cm, minimum width=11.68cm]
			    {Возврат text};

                \node (stop) [terminator, below of=return, yshift=-0.8cm, minimum width=11.68cm]
				{Конец};

				\path [connector] (start) -- (main);
				\path [connector] (main) -- (return);
				\path [connector] (return) -- (stop);
			\end{tikzpicture}
		\end{center}
	\end{item}
	\newpage
	\begin{item}
		Код \textit{алгоритма} на языке \textit{Python}:
		\begin{mycode}
def CorrectionText(text: str) -> str:
    # Убираем лишние пробелы после знаков препинания .,;:!?)]}
    text = re.sub(r'\s+([.,;:!?)}\]])', r'\1', text)

    # Добавляем недостающие пробелы после знаков препинания .,;:!?)]}
    text = re.sub(r'([.,;:!?)}\]])([^\s.,;:!?)}\]])', r'\1 \2', text)

    # Убираем пробелы после открывающих скобок
    text = re.sub(r'(\(\s+)', '(', text)
    text = re.sub(r'(\[\s+)', '[', text)
    text = re.sub(r'(\{\s+)', '{', text)

    # Убираем лишние пробелы перед открывающими скобками
    text = re.sub(r'\s+([\(\[\{])', r' \1', text)

    return text
		\end{mycode}
	\end{item}
	\begin{item}
		\hfill \textit{Таблица 10}

		\centering\textbf{Тестовые данные}

		\begin{table}[h]
			\begin{center}
				\begin{large}
					\begin{tabularx}{\textwidth}{>{\vspace{1pt}}X<{\vspace{4pt}}|>{\vspace{1pt}}X<{\vspace{4pt}}}
						\hline
						Входные данные & Выходные данные \\ \hline
						\makecell[l]{( Привет, Андрей ) , - говорил он} & \makecell[l]{(Привет, Андрей), - говорил он} \\ \hline
						\makecell[l]{(Да $\setminus$n? Ты в этом уверен ? )} & \makecell[l]{(Да? Ты в этом уверен?)} \\ \hline
						\makecell[l]{( ( Проверю на скобках ) )} & \makecell[l]{((Проверю на скобках))} \\ \hline
						\makecell[l]{Да , забавно (получается)} & \makecell[l]{Да, забавно (получается)} \\ \hline
						\makecell[l]{( Чек ) , посмотрим} & \makecell[l]{(Чек), посмотрим} \\ \hline
					\end{tabularx}
				\end{large}
			\end{center}
		\end{table}
	\end{item}
\end{enumerate}