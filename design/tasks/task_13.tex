\section{Тема 13. Графическое решение систем уравнений. Задание 3.}
\label{sec:task13}

\begin{enumerate}
	\item\large
	Найдите точку равновесия в заданном диапазоне с заданным шагом.

		\[
			\begin{cases}
			    y = \frac{2}{x} & \text{в диапазоне } 0.1 \leq x \leq 4, \text{ с шагом } \Delta = 0.1 \\
			    y^2 = 2x
			\end{cases}
		\]
	\begin{item}
		\largeПостроим таблицу данных (x; y). Значения аргумента заполним в диапазоне от 0,1 до 4 с шагом 0,1. В ячейку $B3$ вводим формулу вида = $\frac{2}{x}$, для нахождения значений первой функции. Автозаполнением получаем оставшиеся значения функции. Вторая функция будет иметь значения с обоими знаками, так как \textit{y} в квадрате. Так же запишем значения аргумента в диапазоне от 0,1 до 4 с шагом 0,1, а в соседних колонках формулы $(2 * x) ^ \frac{1}{2}$ и $-((2 * x) ^ \frac{1}{2})$ соответственно, для нахождения значений с противоположными знаками.

		\largeВ результате получена таблица, представленная на рисунке 5.
		\includeimage{assets/task_13_table.png}{0.55}{Таблица значений функций}

        \newpage

		\largeДля построения графиков выберем тип диаграммы -- \textit{График}. Полученный график представлен на рисунке 6.

		\includeimage{assets/task_13_graph.png}{0.95}{Графики функций $y = \frac{2}{x}$ и $y^2 = 2x$ в диапазоне $[0,1;4]$}

		\largeПо графику мы приблизительно можем определить единственную точку пересечения функций -- (1,26; 1,58) с небольшой погрешностью. Для более точных значений необходимо либо уменьшить шаг, либо использовать другие средства поиска точки равновесия.
	\end{item}
\end{enumerate}