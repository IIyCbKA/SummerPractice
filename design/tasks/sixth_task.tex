\section{Тема 6. Линейный поиск. Задание 23.}
\label{sec:task6}

\begin{enumerate}
	\item\largeВ массиве \textit{P(n)} найти самую длинную последовательность, которая является арифметической или геометрической прогрессией.
	\begin{item}
		Словесное описание \textit{алгоритма}:

		\begin{enumerate}
			\item\largeТак как любые два числа образуют прогрессию, получаем коэффициент для поиска прогрессии от них.
			\item\largeИдем по элементам до тех пор, пока соблюдаются условия прогрессии.
			\item\largeКогда прогрессия прерывается, берем ее последние два числа, получаем новый коэффициент и ищем новую прогрессию.
		\end{enumerate}
	\end{item}
	\begin{item}
		Спецификация функции \textit{SearchLargestArithOrGeomProgression}:
		\begin{enumerate}
			\itemЗаголовок: \colorbox{pybg}{\begin{tabular}{l}\textit{def} SearchLargestArithOrGeomProgression(lengthArray: int, \\ array: list) -> int\end{tabular}}
			\itemНазначение: используется для нахождения наибольшей длины алгоритмической или геометрической прогрессии в массиве \textit{P(n)}.
		\end{enumerate}
		\newpage
		Блок-схемы:
		\begin{center}
			\begin{tikzpicture}[node distance=1.2cm]
                \node (start) [terminator]
				{\textit{def} SearchLargestArithOrGeomProgression(lengthArray: int, array: list)};

				\node (if)[decision, below of=start, yshift=-2.2cm]
                {\begin{tabular}{c}lengthArray \\ $\leq$ \\ 2\end{tabular}};

                \node (if_true)[process, below right = 0.7cm of if]
                {largestProgression := lengthArray};

				\node (if_false1)[process, below left = 0.7cm of if, minimum width=6.4cm]
				{\begin{tabular}{c}largestArith := \\ SearchLargestProgression(\\ array, lengthArray, True)\end{tabular}};

				\node (if_false2)[process, below of=if_false1, yshift=-1.5cm, minimum width=6.4cm]
				{\begin{tabular}{c}largestGeom := \\ SearchLargestProgression(\\ array, lengthArray, False)\end{tabular}};

				\node (if_false3)[process, below of=if_false2, yshift=-2.4cm]
				{\begin{tabular}{c}Используем функцию max() \\ для определения наибольшего \\ числа между largestArith и \\ largestGeom. Результат \\ записываем в переменную \\ \textbf{largestProgression}\end{tabular}};

                \node (return) [data, below of=if, yshift=-12cm]
				{Возврат largestProgression};

				\node (stop) [terminator, below of=return, minimum width=13.42cm]
				{Конец};

                \node at($(if) - (3.6, -0.4)$){\bf--};
				\node at($(if) - (-3.6, -0.4)$){\bf+};

                \path [connector] (start) -- (if);
                \path [connector] (if.east) -| (if_true.north);
                \path [connector] (if.west) -| (if_false1.north);
                \path [connector] (if_false1) -- (if_false2);
                \path [connector] (if_false2) -- (if_false3);
                \path [connector] (if_false3.south) -- ++(0, -1) -| (return);
                \path [connector] (if_true.south) -- ++(0, -9.518) -| (return);
                \path [connector] (return) -- (stop);
			\end{tikzpicture}

			\newpage

			\begin{tikzpicture}[node distance=1.2cm]
			    \node (start) [terminator, minimum width=15.1cm]
				{\textit{def} SearchLargestProgression(array: list, length: int, isArith: bool)};

				\node (actionSymbol)[process, below of=start, yshift=-0.3cm, minimum width=15.1cm]
				{\begin{tabular}{c}Если \textbf{isArith}, тогда результатом будет «-», иначе - «/». \\ Результат запишем в переменную \textbf{actionSymbol}\end{tabular}};

                \node (largest)[process, below of=actionSymbol, yshift=-0.3cm, minimum width=15.1cm]
				{largest := 0};

                \node (currentLength)[process, below of=largest, minimum width=15.1cm]
				{currentLength := 2};

				\node (currentCoef)[process, below of=currentLength, yshift=-0.3cm, minimum width=15.1cm]
				{\begin{tabular}{c}Воспользуемся eval() и получим коэффициент между вторым и \\ первым элементами. Результат запишем в переменную \textbf{currentCoef}\end{tabular}};

                \node (for)[cycle, below of=currentCoef, yshift=-0.3cm]
				{$ind := 2 \dots length$};

				\node (actuallyCoef)[process, below of=for, yshift=-0.3cm]
				{\begin{tabular}{c}Воспользуемся eval() и получим коэффициент между a(ind) и \\ a(ind - 1). Результат запишем в переменную \textbf{actuallyCoef}\end{tabular}};

                \node (if)[decision, below of=actuallyCoef, yshift=-2.4cm]
                {\begin{tabular}{c}actuallyCoef \\ = \\ currentCoef\end{tabular}};

                \node (if_true)[process, below right = 0.1cm of if]
                {currentLength := currentLength + 1};

                \node (if_false1)[process, below left = 0.1cm of if]
				{\begin{tabular}{c}Воспользуемся max() для \\ нахождения большего \\ числа между largest и \\ currentLength. Результат \\ запишем в largest\end{tabular}};

				\node (if_false2)[process, below of=if_false1, yshift=-1.2cm, minimum width=5.55cm]
				{currentLength := 2};

				\node (if_false3)[process, below of=if_false2, minimum width=5.55cm]
				{currentCoef := actuallyCoef};

				\node (last_check)[process, below of=if, yshift=-7.5cm]
				{Воспользуемся max() для largest и currentLength. Результат запишем в largest};

				\node (return) [data, below of=last_check, yshift=0.1cm]
				{Возврат largest};

				\node at($(if) - (3.25, -0.4)$){\bf--};
				\node at($(if) - (-3.6, -0.4)$){\bf+};
				\node at($(for) - (-5.25, -0.3)$){Конец цикла};

			    \path [connector] (start) -- (actionSymbol);
			    \path [connector] (actionSymbol) -- (largest);
			    \path [connector] (largest) -- (currentLength);
			    \path [connector] (currentLength) -- (currentCoef);
			    \path [connector] (currentCoef) -- (for);
			    \path [connector] (for) -- (actuallyCoef);
			    \path [connector] (actuallyCoef) -- (if);
			    \path [connector] (if) -| (if_true);
			    \path [connector] (if) -| (if_false1);
			    \path [connector] (if_false1) -- (if_false2);
			    \path [connector] (if_false2) -- (if_false3);
			    \path [connector] (if_false3.south) -- ++(0, -0.3) -- ++(4.15, 0) -- ++(0, -0.5) -- ++(-7.1, 0) |- (for);
			    \path [connector] (if_true.south) -- ++(0, -5.162) -- ++(-4.78, 0) -- ++(0, -0.5) -- ++(-7.1, 0) |- (for);
			    \path [connector] (for.east) -- ++(6.3, 0) -- ++ (0, -13.1) -| (last_check.north);
			    \path [connector] (last_check) -- (return);
			\end{tikzpicture}
		\end{center}
	\end{item}
	\newpage
	\begin{item}
		Код \textit{алгоритма} на языке \textit{Python}:
		\begin{mycode}
def SearchLargestArithOrGeomProgression(lengthArray: int, array: list) -> int:
    if lengthArray <= 2:
        largestProgression: int = lengthArray
    else:
        largestArith: int = SearchLargestProgression(
            array, lengthArray, True)
        largestGeom: int = SearchLargestProgression(
            array, lengthArray, False)
        largestProgression: int = max(largestArith, largestGeom)

    return largestProgression


def SearchLargestProgression(array: list, length: int, isArith: bool) -> int:
    actionSymbol: str = '-' if isArith else '/'
    largest: int = 0
    currentLength: int = 2
    currentCoef: float = eval(f'{array[1]} {actionSymbol} {array[0]}')
    for ind in range(2, length):
        actuallyCoef = eval(f'{array[ind]} {actionSymbol} {array[ind - 1]}')

        if actuallyCoef == currentCoef:
            currentLength += 1
        else:
            largest = max(largest, currentLength)
            currentLength = 2
            currentCoef = actuallyCoef

    largest = max(largest, currentLength)

    return largest
		\end{mycode}
	\end{item}
	\newpage
	\begin{item}
		\hfill \textit{Таблица 6}

		\centering\textbf{Тестовые данные}

		\begin{table}[h]
			\begin{center}
				\begin{large}
					\begin{tabularx}{\textwidth}{>{\vspace{1pt}}X<{\vspace{4pt}}|>{\vspace{1pt}}X<{\vspace{4pt}}}
						\hline
						Входные данные & Выходные данные \\ \hline
						\makecell[l]{\begin{tabular}{l}5 \\ 1 3 5 7 9\end{tabular}} & \makecell[l]{5} \\ \hline
						\makecell[l]{\begin{tabular}{l}5 \\ 2 1 3 5 9\end{tabular}} & \makecell[l]{3} \\ \hline
						\makecell[l]{\begin{tabular}{l}7 \\ 2 4 1 3 9 27 7\end{tabular}} & \makecell[l]{4} \\ \hline
						\makecell[l]{\begin{tabular}{l}4 \\ 3 4 11 2\end{tabular}} & \makecell[l]{2} \\ \hline
						\makecell[l]{\begin{tabular}{l}5 \\ 1 2 4 8 16\end{tabular}} & \makecell[l]{5} \\ \hline
						\makecell[l]{\begin{tabular}{l}2 \\ 1 12\end{tabular}} & \makecell[l]{2} \\ \hline
						\makecell[l]{\begin{tabular}{l}6 \\ 8 9 10 12 14 16\end{tabular}} & \makecell[l]{4} \\ \hline
					\end{tabularx}
				\end{large}
			\end{center}
		\end{table}
	\end{item}
\end{enumerate}