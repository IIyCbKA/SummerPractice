\section{Тема 5. Векторы и матрицы. Задание 23.}
\label{sec:task5}

\begin{enumerate}
	\item\largeМногочлены $P_n(x)$ и $Q_m(x)$ заданы своими коэффициентами. Определить коэффициенты их композиции — многочлена $P_n(Q_m(x))$.
	\begin{item}
		Словесное описание \textit{алгоритма}:

		\begin{enumerate}
			\item\largeДля начала, чтобы получить коэффициенты композиции $P_n(Q_m(x))$, нам необходимо подставить на место \textit{x} в многочлен $P_n(x)$ многочлен $Q_m(x)$.
			\item\largeПосле замены переменной \textit{x} нам необходимо лишь выполнить вычисления для упрощения многочлена.
			\item\largeВышесказанного достаточно, чтобы получить композицию многочленов и выразить ее коэффициенты. Существуют два частных случая:
			    \begin{enumerate}[label=\arabic*.]
			        \item\largeЕсли многочлен $P_n(x)$ нулевой - тогда и композиция будет нулевой.
			        \item\largeЕсли многочлен $Q_m(x)$ нулевой - тогда у композиции будет один коэффициент, равный свободному члену многочлена $P_n(x)$.
			    \end{enumerate}
		\end{enumerate}
	\end{item}
	\begin{item}
		Спецификация функции \textit{ComposePolynomss}:
		\begin{enumerate}
			\itemЗаголовок: \colorbox{pybg}{\begin{tabular}{l}\textit{def} ComposePolynoms(coefsP: list | None,\\ coefsQ: list | None) -> list | None\end{tabular}}
			\itemНазначение: используется для нахождения коэффициентов композиции двух многочленов.
		\end{enumerate}
		\newpage
		Блок-схема:
		\begin{center}
			\begin{tikzpicture}[node distance=1.2cm]
                \node (start) [terminator, minimum width=14cm]
				{\textit{def} ComposePolynoms(coefsP: list | None, coefsQ: list | None)};

				\node (check_P) [process, below of=start, yshift=-0.5cm, minimum width=14cm]
				{\begin{tabular}{c}Если многочлен $P_n(x)$ нулевой, тогда запишем в переменную \textbf{result} то, \\ что коэффициенты композиции нулевые (отметим это None)\end{tabular}};

                \node (check_Q) [process, below of=check_P, yshift=-1.2cm, minimum width=14cm]
				{\begin{tabular}{c}Если многочлен $Q_m(x)$ нулевой, а $P_n(x)$ не нулевой, тогда \\ запишем в переменную \textbf{result} то, что коэффициентом \\ композиции будет свободный член многочлена $P_n(x)$\end{tabular}};

                \node (other) [process, below of=check_Q, yshift=-1.8cm, minimum width=14cm]
				{\begin{tabular}{c}Если оба многочлена не нулевые, тогда воспользуемся библиотекой \\ numpy для нахождения коэффициентов композиции многочленов \\ $P_n(x)$ и $Q_m(x)$, после чего запишем полученный результат \\ в переменную \textbf{result}\end{tabular}};

                \node (return) [data, below of=other, yshift=-1.9cm, minimum width=14cm]
				{Возврат result};

                \node (stop) [terminator, below of=return, yshift=-1cm, minimum width=14cm]
				{Конец};

                \path [connector] (start) -- (check_P);
                \path [connector] (check_P) -- (check_Q);
                \path [connector] (check_Q) -- (other);
                \path [connector] (other) -- (return);
                \path [connector] (return) -- (stop);
			\end{tikzpicture}
		\end{center}
	\end{item}
	\begin{item}
		Код \textit{алгоритма} на языке \textit{Python}:
		\begin{mycode}
def ComposePolynoms(coefsP: list | None, coefsQ: list | None) -> list | None:
    if coefsP is None:
        result: None = None
    elif coefsQ is None:
        result: list = [coefsP[-1]]
    else:
        PolynomP = np.poly1d(coefsP)
        PolynomQ = np.poly1d(coefsQ)
        composedPolynom = PolynomP(PolynomQ)
        result: list = composedPolynom.coefficients.tolist()

    return result
		\end{mycode}
	\end{item}
	\newpage
	\begin{item}
		\hfill \textit{Таблица 5}

		\centering\textbf{Тестовые данные}

		\begin{table}[h]
			\begin{center}
				\begin{large}
					\begin{tabularx}{\textwidth}{>{\vspace{1pt}}X<{\vspace{4pt}}|>{\vspace{1pt}}X<{\vspace{4pt}}}
						\hline
						Входные данные & Выходные данные \\ \hline
						\makecell[l]{\begin{tabular}{l}2 0 -5 \\ 1 -1 2\end{tabular}} & \makecell[l]{2 -4 10 -8 3} \\ \hline
						\makecell[l]{\begin{tabular}{l}1 2 \\ 2 -5\end{tabular}} & \makecell[l]{2 -3} \\ \hline
						\makecell[l]{\begin{tabular}{l}1 \\ 2 0 11\end{tabular}} & \makecell[l]{1} \\ \hline
						\makecell[l]{\begin{tabular}{l}None \\ 2 0 11\end{tabular}} & \makecell[l]{None} \\ \hline
						\makecell[l]{\begin{tabular}{l}2 0 11 \\ None \end{tabular}} & \makecell[l]{11} \\ \hline
					\end{tabularx}
				\end{large}
			\end{center}
		\end{table}
	\end{item}
\end{enumerate}