\section{Тема 9. Линейная алгебра и сжатие информации. Задание 23.}
\label{sec:task9}

\begin{enumerate}
	\item\largeЗаданный неупакованный двоичный массив сжать, используя полубайтовое представление длин цепочек.
	\begin{item}
		Словесное описание \textit{алгоритма}:

		\begin{enumerate}
			\item\largeЗадача довольно непростая, так как подобные способы сжатия практически не используются. Разработаем для этого свой алгоритм.
			\item\largeБудем отталкиваться, что полбайта - это 4 бит. Максимальное число - 15.
			\item\largeСчитаем, что массив начинается с 1, то есть нет незначащих нулей.
			\item\largeБудем хранить в сжатом виде лишь количество подряд идущих бит, то есть цепочки.
			\item\largeПридерживаться будем следующих правил:
			    \begin{enumerate}[label=\arabic*.]
			    \item\largeЕсли длина цепочки меньше 15, тогда записываем фактическую длину цепочки.
			    \item\largeЕсли длина цепочки равна 15, тогда запишем в сжатый массив два числа - 15 и 0. 0 будет признаком того, что длина цепочки кратна 15. Дальше может идти лишь противоположный бит исходного массива.
			    \item\largeЕсли длина цепочки больше 15, тогда делаем переносы, записывая длину числами, не превышающими 15, до полного разложения исходной длины. Если после какого-то числа 15 идет не 0, значит это число также относится к цепочке. На примере: имеем 31 подряд идущий бит 0 в исходном массиве, а затем 3 подряд идущие 1. Запишем это так: [15, 15, 1, 3]. А если в исходном массиве будет 30 подряд идущих бит 0, а затем 3 подряд идущие 1, тогда запишем это так: [15, 15, 0, 3].
			    \end{enumerate}
		\end{enumerate}
	\end{item}
	\begin{item}
		Спецификация функции \textit{BinaryArrayCompression}:
		\begin{enumerate}
			\itemЗаголовок: \colorbox{pybg}{\textit{def} BinaryArrayCompression(array: list) -> list}
			\itemНазначение: используется для сжатия исходного неупакованного двоичного массива, применяя способ полубайтового представления длин цепочек.
		\end{enumerate}

		\newpage

		Блок-схемы:
		\begin{center}
			\begin{tikzpicture}[node distance=1.2cm]
                \node (start) [terminator, minimum width=8.772cm]
			    {\textit{def} BinaryArrayCompression(array: list)};

                \node (init_currentBit)[process, below of=start, minimum width=8.772cm]
				{currentBit := NULL};

				\node (init_currentBitCount)[process, below of=init_currentBit, minimum width=8.772cm]
				{currentBitCount := 0};

				\node (init_compressionArray)[process, below of=init_currentBitCount]
				{Создаем сжатый массив \textbf{compressionArray}};

				\node (for)[cycle, below of=init_compressionArray]
				{$bit := array(0) \dots array(n)$};

                \node (bit_action)[process, below of=for, yshift=-3.7cm]
				{\begin{tabular}{c}Если $currentBit = NULL$, тогда \\ запишем в нее \textit{bit} и добавим к \\ \textbf{currentBitCount} единицу. Если \\ $currentBit = bit$, тогда добавим \\ к \textbf{currentBitCount} единицу. Если \\ $currentBit \neq bit$, тогда вызовем \\ \textit{addChainLengthToCompressedArray}, \\ которая запишет длину \\ цепочки в сжатый массив, \\ сбросит \textbf{currentBitCount} до 1 \\ и запишет \textit{bit} в \textbf{currentBit}\end{tabular}};

                \node (last_signal)[process, below of=bit_action, yshift=-5.4cm, minimum width=8.772cm]
				{\begin{tabular}{c}Вызываем после цикла еще раз \\ \textit{addChainLengthToCompressedArray}, \\ чтобы занести в сжатый массив \\ длину последней цепочки, \\ которая не сбросилась\end{tabular}};

                \node (return) [data, below of=last_signal, yshift=-1.3cm, minimum width=8.772cm]
			    {Возврат compressionArray};

                \node (stop) [terminator, below of=return, minimum width=8.772cm]
				{Конец};

				\node at($(for) - (-4.2, -0.3)$){Конец цикла};

				\path [connector] (start) -- (init_currentBit);
				\path [connector] (init_currentBit) -- (init_currentBitCount);
				\path [connector] (init_currentBitCount) -- (init_compressionArray);
				\path [connector] (init_compressionArray) -- (for);
				\path [connector] (for) -- (bit_action);
				\path [connector] (bit_action.south) -- ++(0, -0.5) -- ++(-4.25, 0) |- (for.west);
				\path [connector] (for.east) -- ++(1.25, 0) -- ++(0, -9.2) -| (last_signal.north);
				\path [connector] (last_signal) -- (return);
				\path [connector] (return) -- (stop);
			\end{tikzpicture}

			\newpage

			\begin{tikzpicture}[node distance=1.2cm]
			    \node (start) [terminator]
			    {\begin{tabular}{l}\textit{def} addChainLengthToCompressedArray( \\ chainLength: int, compressedArray: list)\end{tabular}};

                \node (if1)[decision, below of=start, yshift=-2.2cm]
                {\begin{tabular}{c}chainLength \\ < 15\end{tabular}};

                \node (if1_true)[process, below right = 0.5cm of if1]
			    {\begin{tabular}{c}Запишем длину \\ цепочки в массив \\ \textbf{compressedArray}\end{tabular}};

                \node (if2)[decision, below left = 0.5cm of if1, xshift=-0.7cm]
                {\begin{tabular}{c}chainLength \\ = 15\end{tabular}};

                \node (if2_true)[process, below right = 0.4cm of if2]
			    {\begin{tabular}{c}Запишем в массив \\ \textbf{compressedArray} \\ 15 и 0\end{tabular}};

			    \node (if2_false)[process, below left = 0.4cm of if2]
			    {\begin{tabular}{c}Записываем в массив \\ \textbf{compressedArray} \\ длину цепочки \\ числами, не \\ превышающими 15, \\ раскладывая длину\end{tabular}};

                \node (stop) [terminator, below of=if1, yshift=-9cm, minimum width=8.3cm]
				{Конец};

				\node at($(if1) - (-2.9, -0.4)$){\bf+};
				\node at($(if1) - (2.7, -0.4)$){\bf--};

				\node at($(if2) - (-2.7, -0.4)$){\bf+};
				\node at($(if2) - (2.9, -0.4)$){\bf--};

                \path [connector] (start) -- (if1);
                \path [connector] (if1) -| (if1_true);
                \path [connector] (if1) -| (if2);
                \path [connector] (if2) -| (if2_true);
                \path [connector] (if2) -| (if2_false);
                \path [connector] (if1_true.south) -- ++(0, -5.85) -| (stop.north);
                \path [connector] (if2_true.south) -- ++(0, -2.8) -- ++(-3.58, 0) -- ++ (0, -0.552) -| (stop.north);
                \path [connector] (if2_false.south) -- ++(0, -0.955) -- ++(3.725, 0) -- ++ (0, -0.552) -| (stop.north);
			\end{tikzpicture}
		\end{center}
	\end{item}
	\newpage
	\begin{item}
		Код \textit{алгоритма} на языке \textit{Python}:
		\begin{mycode}
def BinaryArrayCompression(array: list) -> list:
    currentBit: int | None = None
    currentBitCount: int = 0
    compressionArray: list = []

    for bit in array:
        if currentBit is None:
            currentBit = bit
            currentBitCount += 1
        elif bit == currentBit:
            currentBitCount += 1
        else:
            addChainLengthToCompressedArray(currentBitCount, compressionArray)
            currentBitCount = 1
            currentBit = bit

    if currentBitCount > 0:
        addChainLengthToCompressedArray(currentBitCount, compressionArray)

    return compressionArray


def addChainLengthToCompressedArray(chainLength: int,
                                    compressedArray: list) -> None:
    if chainLength < 15:
        compressedArray.append(chainLength)
    elif chainLength == 15:
        compressedArray.append(chainLength)
        compressedArray.append(0)
    else:
        while chainLength > 0:
            if chainLength >= 15:
                numAdd = 15
            else:
                numAdd = chainLength

            chainLength -= numAdd
            compressedArray.append(numAdd)
		\end{mycode}
	\end{item}
	\newpage
	\begin{item}
		\hfill \textit{Таблица 9}

		\centering\textbf{Тестовые данные}

		\begin{table}[h]
			\begin{center}
				\begin{large}
					\begin{tabularx}{\textwidth}{>{\vspace{1pt}}X<{\vspace{4pt}}|>{\vspace{1pt}}X<{\vspace{4pt}}}
						\hline
						Входные данные & Выходные данные \\ \hline
						\makecell[l]{1 1 1 0 0 1} & \makecell[l]{3 2 1} \\ \hline
						\makecell[l]{1 0 1 0 1 0} & \makecell[l]{1 1 1 1 1 1} \\ \hline
						\makecell[l]{1 1 1 1 1 1 1 1 1 1 1 1 1 1 1 0 0} & \makecell[l]{15 0 2} \\ \hline
						\makecell[l]{1 1 1 1 1 1 1 1 1 1 1 1 1 1 1 1 1 0 1} & \makecell[l]{15 2 1 1} \\ \hline
					\end{tabularx}
				\end{large}
			\end{center}
		\end{table}
	\end{item}
\end{enumerate}