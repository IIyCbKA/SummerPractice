\chapter{Задания к работе}

\begin{center}\largeВариант 23\end{center}

\begin{enumerate}
	\item\largeТекущее время (часы, минуты, секунды) задано тремя переменными: \textit{h, m, s}. Округлить его до целых значений минут и часов. Например, 14 ч 21 мин 45 с преобразуется в 14 ч 22 мин или 14 ч, а 9 ч 59 мин 23 с — соответственно в 9 ч 59 мин или 10 ч.
    \item\largeДля заданного 0 < \textit{n} $\leq$ 200, рассматриваемого как возраст человека, вывести фразу вида: «Мне 21 год», «Мне 32 года», «Мне 12 лет».
    \item\largeЛеспромхоз ведёт заготовку деловой древесины. Первоначальный объём её на территории леспромхоза составлял \textit{p} кубометров. Ежегодный прирост составляет \textit{k}\%. Годовой план заготовки — \textit{t} кубометров. Через сколько лет в бывшем лесу будут расти одни опята?
    \item\largeКаждый из элементов $x_i$ массива \textit{X(n)} заменить средним значением первых \textit{i} элементов этого массива.
    \item\largeМногочлены $P_n(x)$ и $Q_m(x)$ заданы своими коэффициентами. Определить коэффициенты их композиции — многочлена $P_n(Q_m(x))$.
    \item\largeВ массиве \textit{P(n)} найти самую длинную последовательность, которая является арифметической или геометрической прогрессией.
    \item\largeНайти все натуральные числа, не превосходящие заданного \textit{n}, десятичная запись которых есть строго возрастающая или строго убывающая последовательность цифр.
    \item\largeМедианой множества точек на плоскости назовём прямую, которая делит множество на два подмножества одинаковой мощности. Найти горизонтальную и вертикальную медианы заданного множества, у которого никакие две точки не лежат на одной горизонтальной или вертикальной прямой.
    \item\largeЗаданный неупакованный двоичный массив сжать, используя полубайтовое представление длин цепочек.
    \item\largeПо правилам пунктуации пробел может стоять после, а не перед каждым из следующих знаков: . , ; : ! ? ) ] \} …; перед, а не после знаков: ( [ \{. Заданный текст проверить на соблюдение этих правил и при необходимости исправить. Вместо пробела может быть перевод строки или знак табуляции.
    \item\largeПостроить прямую $3x + 2y - 4 = 0$ в диапазоне $x \in [-1;3]$ с шагом $\Delta = 0,25$
    \item\largeПостроить верхнюю часть эллипса $0,1 \leq x \leq 5,1$ с шагом $\Delta = 0,25$, заданного уравнением $\frac{x^2}{4} + y^2 = 1$
    \begin{item}
		\largeНайдите точку равновесия в заданном диапазоне с заданным шагом.

		\[
			\begin{cases}
			    y = \frac{2}{x} & \text{в диапазоне } 0.1 \leq x \leq 4, \text{ с шагом } \Delta = 0.1 \\
			    y^2 = 2x
			\end{cases}
		\]
	\end{item}
    \item\largeПостроить плоскость, проходящую через точки $M_1$(3,3,1), $M_2$(2,3,2), $M_3$(1,1,3), при $-1 \leq x \leq 4$ с шагом $\Delta = 0,5$ и $-1 \leq y \leq 3$ с шагом $\Delta = 1$.
    \item\largeПостроить часть параболоида, заданного уравнением $\frac{x^2}{9} + \frac{y^2}{4} = 2z$, лежащую в диапазоне $-3 \leq x \leq 3$, $-2 \leq y \leq 2$ с шагом $\Delta = 0,5$ для обеих переменных.
\end{enumerate}