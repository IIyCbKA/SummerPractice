\chapter{Заключение}

\largeВ ходе выполнения компьютерной практики были получены навыки работы с LaTeX, его формулами, блок-схемами и библиотеками; закреплены навыки построения блок-схем и навыки работы с языком Python.

\chapter{Список используемой литературы}

\begin{enumerate}
	\item\large Tantau T. The TikZ and PGF Packages. Manual for version 2.10 — Institut für Theoretische Informatik, Universität zu Lübeck. — October 25, 2010 (http://sourceforge.net/projects/pgf).
	\item\large D. E. Knuth: The TEXbook, часть A серии Computers and Typesetting, Addison-Wesley Publishing Company (1984), ISBN 0-201-13448-9. Русский перевод: Дональд Е. Кнут. Все про TEX. М., 1993.
	\item\large Притчин, И. С. Основы программирования : учеб. пособие / И. С. Притчин ; БГТУ им. В. Г. Шухова, Каф. програм. обеспечения вычисл. техники и автоматизир. систем. -- Электрон. текстовые дан. - Белгород : Изд-во БГТУ им. В. Г. Шухова, 2022. - Б.ц. Э.Р. N 6551.
	\item\largeЛьвовский, С. М. Работа в системе LaTeX : учебное пособие / С. М. Львовский ; Национальный Открытый Университет "ИНТУИТ". - Москва : Интернет-Университет Информационных Технологий (ИНТУИТ), 2007. - 465 с.
\end{enumerate}
