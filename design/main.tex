\documentclass{report}

% Подключаем необходимые библиотеки. %
\usepackage[russian]{babel}
\usepackage[utf8]{inputenc}
\usepackage{geometry}
\usepackage{minted}
\usepackage[hidelinks]{hyperref}
\usepackage{xcolor}
\usepackage{xspace}
\usepackage{fancyhdr}
\usepackage{tocloft}
\usepackage{booktabs}
\usepackage{tikz}
\usepackage{titlesec}
\usepackage{tabularx}
\usepackage{setspace}
\usepackage{makecell}
\usepackage{float}
\usepackage{amsmath}

\counterwithout{figure}{chapter}

\usetikzlibrary{shapes, arrows}
\usetikzlibrary{positioning}
\usetikzlibrary{shapes.geometric}
\usetikzlibrary{shapes.misc}
\usetikzlibrary{calc}
\usetikzlibrary{chains}

% Предопределяем типы блок-схем.
\tikzstyle{terminator} = [rectangle, draw, text centered, rounded corners, minimum height=2em]
\tikzstyle{process} = [rectangle, draw, text centered, minimum height=2em]
\tikzstyle{decision} = [diamond, draw, text centered, minimum height=2em]
\tikzstyle{data}=[trapezium, draw, text centered, trapezium left angle=60, trapezium right angle=120, minimum height=2em]
\tikzstyle{cycle}=[chamfered rectangle, draw, text centered,chamfered rectangle xsep=10pt ]
\tikzstyle{connector} = [draw, -latex']

% Убираем отступ надписи <<Оглавление>>. %
\setlength{\cftbeforetoctitleskip}{0pt}
\setlength{\cftaftertoctitleskip}{2pt}
% Форматируем надпись оглавление. %
\renewcommand{\cfttoctitlefont}{\LARGE\bfseries}

\titleformat{\chapter}{\normalfont\LARGE\bfseries}{\thechapter}{20pt}{\LARGE\bfseries}
\titlespacing*{\chapter}{0pt}{0pt}{20pt}

% Настраиваем колонтикулы. %
\pagestyle{fancy}
% Очистить текущие заголовки и нижние колонтитулы. %
\fancyhf{}
% Номер страницы в центре нижнего колонтитула. %
\fancyfoot[C]{\thepage}
% Убрать линию между заголовком и текстом. %
\renewcommand{\headrulewidth}{0pt}
% Убрать линию между текстом и нижним колонтитулом. %
\renewcommand{\footrulewidth}{0pt}

% Создаем цвета. %
\definecolor{pybg}{rgb}{0.95,0.95,0.95}

% Устанавливаем необходимые поля. %
\geometry{left=3cm}
\geometry{right=1.5cm}
\geometry{top=2cm}
\geometry{bottom=2cm}

% Сетапим форматирование кода в документе. %
\newminted[mycode]{python}{
	bgcolor=pybg,
	baselinestretch=1.2,
	fontsize=\normalsize,
	tabsize=0,
	linenos
}

% Создадим команду для отрисовки изображения. %
% \newimage{имя_файла}{ширина}{заголовок_картинки}
\newcommand{\includeimage}[3]{
	\begin{figure}[H]
		%\begin{center}
		\centering
			\includegraphics[width=#2\textwidth]{#1}
			\caption{#3}
		%\end{center}
	\end{figure}
}

\begin{document}
	\thispagestyle{empty}

	\begin{center}
	МИНИСТЕРСТВО НАУКИ И ВЫСШЕГО ОБРАЗОВАНИЯ

	РОССИЙСКОЙ ФЕДЕРАЦИИ

	\vskip 0.5cm

	ФЕДЕРАЛЬНОЕ ГОСУДАРСТВЕННОЕ БЮДЖЕТНОЕ

	ОБРАЗОВАТЕЛЬНОЕ УЧРЕЖДЕНИЕ ВЫСШЕГО ОБРАЗОВАНИЯ

	\vskip 0.5cm

	\textbf{<<БЕЛГОРОДСКИЙ ГОСУДАРСТВЕННЫЙ}

	\textbf{ТЕХНОЛОГИЧЕСКИЙ УНИВЕРСИТЕТ им. В.Г.ШУХОВА>>}

	\textbf{(БГТУ им. В.Г. Шухова)}

	\vskip 0.5cm

	Кафедра программного обеспечения вычислительной техники и

	автоматизированных систем

	\vskip 3cm

	\large{\bfКомпьютерная практика}

	\vskip 7cm

	\begin{flushright}
		Выполнил: ст. группы ПВ-232

		Чернобровнеко А.Е.

		Проверил: Солонченко Р. Е.
	\end{flushright}

	\vskip 6cm

	Белгород, 2024г.
\end{center}

\newpage

\onehalfspacing\tableofcontents

\newpage


	\begin{onehalfspacing}
		\chapter{Задания к работе}

\begin{center}\largeВариант 23\end{center}

\begin{enumerate}
	\item\largeТекущее время (часы, минуты, секунды) задано тремя переменными: \textit{h, m, s}. Округлить его до целых значений минут и часов. Например, 14 ч 21 мин 45 с преобразуется в 14 ч 22 мин или 14 ч, а 9 ч 59 мин 23 с — соответственно в 9 ч 59 мин или 10 ч.
    \item\largeДля заданного 0 < \textit{n} $\leq$ 200, рассматриваемого как возраст человека, вывести фразу вида: «Мне 21 год», «Мне 32 года», «Мне 12 лет».
    \item\largeЛеспромхоз ведёт заготовку деловой древесины. Первоначальный объём её на территории леспромхоза составлял \textit{p} кубометров. Ежегодный прирост составляет \textit{k}\%. Годовой план заготовки — \textit{t} кубометров. Через сколько лет в бывшем лесу будут расти одни опята?
    \item\largeКаждый из элементов $x_i$ массива \textit{X(n)} заменить средним значением первых \textit{i} элементов этого массива.
    \item\largeМногочлены $P_n(x)$ и $Q_m(x)$ заданы своими коэффициентами. Определить коэффициенты их композиции — многочлена $P_n(Q_m(x))$.
    \item\largeВ массиве \textit{P(n)} найти самую длинную последовательность, которая является арифметической или геометрической прогрессией.
    \item\largeНайти все натуральные числа, не превосходящие заданного \textit{n}, десятичная запись которых есть строго возрастающая или строго убывающая последовательность цифр.
    \item\largeМедианой множества точек на плоскости назовём прямую, которая делит множество на два подмножества одинаковой мощности. Найти горизонтальную и вертикальную медианы заданного множества, у которого никакие две точки не лежат на одной горизонтальной или вертикальной прямой.
    \item\largeЗаданный неупакованный двоичный массив сжать, используя полубайтовое представление длин цепочек.
    \item\largeПо правилам пунктуации пробел может стоять после, а не перед каждым из следующих знаков: . , ; : ! ? ) ] \} …; перед, а не после знаков: ( [ \{. Заданный текст проверить на соблюдение этих правил и при необходимости исправить. Вместо пробела может быть перевод строки или знак табуляции.
    \item\largeПостроить прямую $3x + 2y - 4 = 0$ в диапазоне $x \in [-1;3]$ с шагом $\Delta = 0,25$
    \item\largeПостроить верхнюю часть эллипса $0,1 \leq x \leq 5,1$ с шагом $\Delta = 0,25$, заданного уравнением $\frac{x^2}{4} + y^2 = 1$
    \begin{item}
		\largeНайдите точку равновесия в заданном диапазоне с заданным шагом.

		\[
			\begin{cases}
			    y = \frac{2}{x} & \text{в диапазоне } 0.1 \leq x \leq 4, \text{ с шагом } \Delta = 0.1 \\
			    y^2 = 2x
			\end{cases}
		\]
	\end{item}
    \item\largeПостроить плоскость, проходящую через точки $M_1$(3,3,1), $M_2$(2,3,2), $M_3$(1,1,3), при $-1 \leq x \leq 4$ с шагом $\Delta = 0,5$ и $-1 \leq y \leq 3$ с шагом $\Delta = 1$.
    \item\largeПостроить часть параболоида, заданного уравнением $\frac{x^2}{9} + \frac{y^2}{4} = 2z$, лежащую в диапазоне $-3 \leq x \leq 3$, $-2 \leq y \leq 2$ с шагом $\Delta = 0,5$ для обеих переменных.
\end{enumerate} \newpage

		\chapter{Основная часть}
\section{Тема 1. Линейные алгоритмы. Задание 23.}
\label{sec:task1}

\begin{enumerate}
	\item\largeТекущее время (часы, минуты, секунды) задано тремя переменными: \textit{h, m, s}. Округлить его до целых значений минут и часов. Например, 14 ч 21 мин 45 с преобразуется в 14 ч 22 мин или 14 ч, а 9 ч 59 мин 23 с — соответственно в 9 ч 59 мин или 10 ч.
	\begin{item}
		Словесное описание \textit{алгоритма}:

		\begin{enumerate}
			\item\largeЗнаем, что в одном часе 60 минут, а в одной минуте 60 секунд. Будем придерживаться принципа округления - если количество секунд $<50\%$, тогда округляем количество минут в меньшую сторону, иначе - в большую. С минутами в часах будем действовать по тому же принципу.
			\item\largeИсходя из вышесказанного, если количество секунд $<30$, тогда округляем минуты в меньшую сторону, иначе - в большую. С минутами в часах действуем по тому же принципу.
			\item\largeДля решения сначала запишем результаты округлений в соответствующие переменные, а потом объединим их в общую.
		\end{enumerate}
	\end{item}
	\begin{item}
		Спецификация функции \textit{TimeRounding}:
		\begin{enumerate}
			\itemЗаголовок: \colorbox{pybg}{\textit{def} TimeRounding(hours: int, minutes: int, seconds: int) -> str:}
			\itemНазначение: используется для нахождения целого числа минут и часов или только часов по введенному времени.
		\end{enumerate}
		\newpage
		Блок-схема:
		\begin{center}
			\begin{tikzpicture}[node distance=1.2cm]
				\node (start) [terminator]
				{\textit{def} TimeRounding(hours: int, minutes: int, seconds: int)};

				\node (compute_only_hours) [process, below of=start, yshift=-0.5cm]
				{\begin{tabular}{c}Приведем переменную \textbf{hours} к строковому виду с округлением по минутам \\ из переменной \textbf{minutes}, с сохранением результата в переменную \textbf{only\_hours}\end{tabular}};

				\node (compute_hours_with_minutes) [process, below of=compute_only_hours, yshift=-1cm]
				{\begin{tabular}{c}Приведем переменные \textbf{hours} и \textbf{minutes} к строковому \\ виду с округлением по секундам из переменной \textbf{seconds}, \\ с сохранением результата в переменную \textbf{hours\_with\_minutes}\end{tabular}};

				\node (compute_result) [process, below of=compute_hours_with_minutes, yshift=-1.2cm]
				{\begin{tabular}{c}Объединим переменные \textbf{only\_hours} и \textbf{hours\_with\_minutes} \\ и запишем результат в переменную \textbf{result}\end{tabular}};

				\node (return) [data, below of=compute_result, yshift=-0.4cm]
				{Возврат result};

				\node (stop) [terminator, below of=return]
				{Конец};

				\path [connector] (start) -- (compute_only_hours);
				\path [connector] (compute_only_hours) -- (compute_hours_with_minutes);
				\path [connector] (compute_hours_with_minutes) -- (compute_result);
				\path [connector] (compute_result) -- (return);
				\path [connector] (return) -- (stop);
			\end{tikzpicture}
		\end{center}
	\end{item}
	\begin{item}
		Код \textit{алгоритма} на языке \textit{Python}:
		\begin{mycode}
def TimeRounding(hours: int, minutes: int, seconds: int) -> str:
    only_hours: str = f'{hours + 1 if minutes >= 30 else hours} ч'
    hours_with_minutes: str = \
        f'{hours} ч {minutes + 1 if seconds >= 30 else minutes} м'
    result: str = f'{hours_with_minutes} или {only_hours}'

    return result
		\end{mycode}
	\end{item}
	\begin{item}
		\hfill \textit{Таблица 1}

		\centering\textbf{Тестовые данные}

		\begin{table}[h]
			\begin{center}
				\begin{large}
					\begin{tabularx}{\textwidth}{>{\vspace{1pt}}X<{\vspace{4pt}}|>{\vspace{1pt}}X<{\vspace{4pt}}}
						\hline
						Входные данные & Выходные данные \\ \hline
						\makecell[l]{\begin{tabular}{c}14 \\ 21 \\ 45\end{tabular}} & \makecell[l]{14 ч 22 м или 14 ч} \\ \hline
						\makecell[l]{\begin{tabular}{c}9 \\ 59 \\ 23\end{tabular}} & \makecell[l]{9 ч 59 м или 10 ч} \\ \hline
					\end{tabularx}
				\end{large}
			\end{center}
		\end{table}
	\end{item}
\end{enumerate} \newpage

		\section{Тема 2. Разветвляющиеся алгоритмы. Задание 23.}
\label{sec:task2}

\begin{enumerate}
	\item\largeДля заданного 0 < \textit{n} $\leq$ 200, рассматриваемого как возраст человека, вывести фразу вида: «Мне 21 год», «Мне 32 года», «Мне 12 лет».
	\begin{item}
		Словесное описание \textit{алгоритма}:

		\begin{enumerate}
			\item\largeЕсли $n \% 10 = 0$, $n \% 10 \geq 5$ или $10 \leq n \leq 20$ - мы пишем «лет». Если \textit{n} оканчивается на 1, при этом $n \neq 11$, тогда мы пишем «год». В остальных случаях мы пишем «года».
			\item\largeИсходя из вышесказанного, запишем условия для получения правильной формы слова.
			\item\largeПолучив правильную форму слова, объединим ее с шаблоном предложения.
		\end{enumerate}
	\end{item}
	\begin{item}
		Спецификация функции \textit{ReturnAgeText}:
		\begin{enumerate}
			\itemЗаголовок: \colorbox{pybg}{\textit{def} ReturnAgeText(age: int) -> str}
			\itemНазначение: используется для нахождения правильной формы слова «лет», обозначающего возраст, в шаблон «Мне \textit{n} лет».
		\end{enumerate}
		\newpage
		Блок-схема:
		\begin{center}
			\begin{tikzpicture}[node distance=1.2cm]
				\node (start) [terminator]
				{\textit{def} ReturnAgeText(age: int)};

                \node (if)[decision, below of=start, node distance = 4cm]
                {\begin{tabular}{c}$0 < age \% 100 < 20$ \\ or $age \% 10 >= 5$ \\ or $age \% 10 = 0$\end{tabular}};

                \node (if_true)[process, below right = 1cm of if]
                {correctForm := «лет»};

				\node (second_if)[decision, below left = 2cm of if, yshift = 1cm, xshift = -0.5cm]
				{$age \% 10 = 1$};

                \node (second_if_true)[process, below right = 0.7cm of second_if]
                {correctForm := «год»};

				\node (second_if_false)[process, below left = 0.7cm of second_if]
				{correctForm := «года»};

				\node (result)[process, below of = if, node distance = 7cm]
				{\begin{tabular}{c}Объединим правильную форму слова с шаблоном, \\ результат запишем в перменную \textbf{result}\end{tabular}};

				\node (return) [data, below of=result, yshift=-0.4cm]
				{Возврат result};

				\node (stop) [terminator, below of=return, minimum width=5.53cm]
				{Конец};

                \node at($(if) - (3.5, -0.4)$){\bf--};
				\node at($(if) - (-3.5, -0.4)$){\bf+};
                \node at($(second_if) - (2.4, -0.4)$){\bf--};
				\node at($(second_if) - (-2.4, -0.4)$){\bf+};

				\path [connector] (start) -- (if);
				\path [connector] (if.east) -- ++(1, 0) -| (if_true.north);
				\path [connector] (if.west) -- ++(-1, 0) -| (second_if.north);
				\path [connector] (second_if.east) -- ++(1.5, 0) -| (second_if_true.north);
				\path [connector] (second_if.west) -- ++(-1.8, 0) -| (second_if_false.north);

                \path [connector] (if_true) -- ++(0, -3) -| (result.north);
                \path [connector] (second_if_false) -- ++(0, -1) -- ++(3.5, 0) -- ++(0, -0.245) -| (result.north);
                \path [connector] (second_if_true) -- ++(0, -1) -- ++(-3.39, 0) -- ++(0, -0.245) -| (result.north);

                \path [connector] (result) -- (return);
                \path [connector] (return) -- (stop);
			\end{tikzpicture}
		\end{center}
	\end{item}
	\begin{item}
		Код \textit{алгоритма} на языке \textit{Python}:
		\begin{mycode}
def ReturnAgeText(age: int) -> str:
    if (10 < age % 100 < 20) or (age % 10 >= 5) or (age % 10 == 0):
        correctForm: str = 'лет'
    elif age % 10 == 1:
        correctForm: str = 'год'
    else:
        correctForm: str = 'года'
    answer: str = f'Мне {age} {correctForm}'

    return answer
		\end{mycode}
	\end{item}
	\newpage
	\begin{item}
		\hfill \textit{Таблица 1}

		\centering\textbf{Тестовые данные}

		\begin{table}[h]
			\begin{center}
				\begin{large}
					\begin{tabularx}{\textwidth}{>{\vspace{1pt}}X<{\vspace{4pt}}|>{\vspace{1pt}}X<{\vspace{4pt}}}
						\hline
						Входные данные & Выходные данные \\ \hline
						\makecell[l]{11} & \makecell[l]{Мне 11 лет} \\ \hline
						\makecell[l]{1} & \makecell[l]{Мне 1 год} \\ \hline
						\makecell[l]{21} & \makecell[l]{Мне 21 год} \\ \hline
						\makecell[l]{30} & \makecell[l]{Мне 30 лет} \\ \hline
						\makecell[l]{42} & \makecell[l]{Мне 42 года} \\ \hline
					\end{tabularx}
				\end{large}
			\end{center}
		\end{table}
	\end{item}
\end{enumerate} \newpage

	\end{onehalfspacing}
\end{document}