\documentclass{report}

% Подключаем необходимые библиотеки. %
\usepackage[russian]{babel}
\usepackage[utf8]{inputenc}
\usepackage{geometry}
\usepackage{minted}
\usepackage[hidelinks]{hyperref}
\usepackage{xcolor}
\usepackage{xspace}
\usepackage{fancyhdr}
\usepackage{tocloft}
\usepackage{booktabs}
\usepackage{tikz}
\usepackage{titlesec}
\usepackage{tabularx}
\usepackage{setspace}
\usepackage{makecell}
\usepackage{float}
\usepackage{amsmath}
\usepackage{enumitem}

\counterwithout{figure}{chapter}

\usetikzlibrary{shapes, arrows}
\usetikzlibrary{positioning}
\usetikzlibrary{shapes.geometric}
\usetikzlibrary{shapes.misc}
\usetikzlibrary{calc}
\usetikzlibrary{chains}

% Предопределяем типы блок-схем.
\tikzstyle{terminator} = [rectangle, draw, text centered, rounded corners, minimum height=2em]
\tikzstyle{process} = [rectangle, draw, text centered, minimum height=2em]
\tikzstyle{decision} = [diamond, draw, text centered, minimum height=2em]
\tikzstyle{data}=[trapezium, draw, text centered, trapezium left angle=60, trapezium right angle=120, minimum height=2em]
\tikzstyle{cycle}=[chamfered rectangle, draw, text centered,chamfered rectangle xsep=10pt ]
\tikzstyle{connector} = [draw, -latex']

% Убираем отступ надписи <<Оглавление>>. %
\setlength{\cftbeforetoctitleskip}{0pt}
\setlength{\cftaftertoctitleskip}{2pt}
% Форматируем надпись оглавление. %
\renewcommand{\cfttoctitlefont}{\LARGE\bfseries}

\titleformat{\chapter}{\normalfont\LARGE\bfseries}{\thechapter}{20pt}{\LARGE\bfseries}
\titlespacing*{\chapter}{0pt}{0pt}{20pt}

% Настраиваем колонтикулы. %
\pagestyle{fancy}
% Очистить текущие заголовки и нижние колонтитулы. %
\fancyhf{}
% Номер страницы в центре нижнего колонтитула. %
\fancyfoot[C]{\thepage}
% Убрать линию между заголовком и текстом. %
\renewcommand{\headrulewidth}{0pt}
% Убрать линию между текстом и нижним колонтитулом. %
\renewcommand{\footrulewidth}{0pt}

% Создаем цвета. %
\definecolor{pybg}{rgb}{0.95,0.95,0.95}

% Устанавливаем необходимые поля. %
\geometry{left=3cm}
\geometry{right=1.5cm}
\geometry{top=2cm}
\geometry{bottom=2cm}

% Сетапим форматирование кода в документе. %
\newminted[mycode]{python}{
	bgcolor=pybg,
	baselinestretch=1.2,
	fontsize=\normalsize,
	tabsize=0,
	linenos
}

% Создадим команду для отрисовки изображения. %
% \newimage{имя_файла}{ширина}{заголовок_картинки}
\newcommand{\includeimage}[3]{
	\begin{figure}[H]
		%\begin{center}
		\centering
			\includegraphics[width=#2\textwidth]{#1}
			\caption{#3}
		%\end{center}
	\end{figure}
}

\begin{document}
	\thispagestyle{empty}

	\begin{center}
	МИНИСТЕРСТВО НАУКИ И ВЫСШЕГО ОБРАЗОВАНИЯ

	РОССИЙСКОЙ ФЕДЕРАЦИИ

	\vskip 0.5cm

	ФЕДЕРАЛЬНОЕ ГОСУДАРСТВЕННОЕ БЮДЖЕТНОЕ

	ОБРАЗОВАТЕЛЬНОЕ УЧРЕЖДЕНИЕ ВЫСШЕГО ОБРАЗОВАНИЯ

	\vskip 0.5cm

	\textbf{<<БЕЛГОРОДСКИЙ ГОСУДАРСТВЕННЫЙ}

	\textbf{ТЕХНОЛОГИЧЕСКИЙ УНИВЕРСИТЕТ им. В.Г.ШУХОВА>>}

	\textbf{(БГТУ им. В.Г. Шухова)}

	\vskip 0.5cm

	Кафедра программного обеспечения вычислительной техники и

	автоматизированных систем

	\vskip 3cm

	\large{\bfКомпьютерная практика}

	\vskip 7cm

	\begin{flushright}
		Выполнил: ст. группы ПВ-232

		Чернобровнеко А.Е.

		Проверил: Солонченко Р. Е.
	\end{flushright}

	\vskip 6cm

	Белгород, 2024г.
\end{center}

\newpage

\onehalfspacing\tableofcontents

\newpage


	\begin{onehalfspacing}
		\chapter{Задания к работе}

\begin{center}\largeВариант 23\end{center}

\begin{enumerate}
	\item\largeТекущее время (часы, минуты, секунды) задано тремя переменными: \textit{h, m, s}. Округлить его до целых значений минут и часов. Например, 14 ч 21 мин 45 с преобразуется в 14 ч 22 мин или 14 ч, а 9 ч 59 мин 23 с — соответственно в 9 ч 59 мин или 10 ч.
    \item\largeДля заданного 0 < \textit{n} $\leq$ 200, рассматриваемого как возраст человека, вывести фразу вида: «Мне 21 год», «Мне 32 года», «Мне 12 лет».
    \item\largeЛеспромхоз ведёт заготовку деловой древесины. Первоначальный объём её на территории леспромхоза составлял \textit{p} кубометров. Ежегодный прирост составляет \textit{k}\%. Годовой план заготовки — \textit{t} кубометров. Через сколько лет в бывшем лесу будут расти одни опята?
    \item\largeКаждый из элементов $x_i$ массива \textit{X(n)} заменить средним значением первых \textit{i} элементов этого массива.
    \item\largeМногочлены $P_n(x)$ и $Q_m(x)$ заданы своими коэффициентами. Определить коэффициенты их композиции — многочлена $P_n(Q_m(x))$.
    \item\largeВ массиве \textit{P(n)} найти самую длинную последовательность, которая является арифметической или геометрической прогрессией.
    \item\largeНайти все натуральные числа, не превосходящие заданного \textit{n}, десятичная запись которых есть строго возрастающая или строго убывающая последовательность цифр.
    \item\largeМедианой множества точек на плоскости назовём прямую, которая делит множество на два подмножества одинаковой мощности. Найти горизонтальную и вертикальную медианы заданного множества, у которого никакие две точки не лежат на одной горизонтальной или вертикальной прямой.
    \item\largeЗаданный неупакованный двоичный массив сжать, используя полубайтовое представление длин цепочек.
    \item\largeПо правилам пунктуации пробел может стоять после, а не перед каждым из следующих знаков: . , ; : ! ? ) ] \} …; перед, а не после знаков: ( [ \{. Заданный текст проверить на соблюдение этих правил и при необходимости исправить. Вместо пробела может быть перевод строки или знак табуляции.
    \item\largeПостроить прямую $3x + 2y - 4 = 0$ в диапазоне $x \in [-1;3]$ с шагом $\Delta = 0,25$
    \item\largeПостроить верхнюю часть эллипса $0,1 \leq x \leq 5,1$ с шагом $\Delta = 0,25$, заданного уравнением $\frac{x^2}{4} + y^2 = 1$
    \begin{item}
		\largeНайдите точку равновесия в заданном диапазоне с заданным шагом.

		\[
			\begin{cases}
			    y = \frac{2}{x} & \text{в диапазоне } 0.1 \leq x \leq 4, \text{ с шагом } \Delta = 0.1 \\
			    y^2 = 2x
			\end{cases}
		\]
	\end{item}
    \item\largeПостроить плоскость, проходящую через точки $M_1$(3,3,1), $M_2$(2,3,2), $M_3$(1,1,3), при $-1 \leq x \leq 4$ с шагом $\Delta = 0,5$ и $-1 \leq y \leq 3$ с шагом $\Delta = 1$.
    \item\largeПостроить часть параболоида, заданного уравнением $\frac{x^2}{9} + \frac{y^2}{4} = 2z$, лежащую в диапазоне $-3 \leq x \leq 3$, $-2 \leq y \leq 2$ с шагом $\Delta = 0,5$ для обеих переменных.
\end{enumerate} \newpage

		\chapter{Основная часть}
\section{Тема 1. Линейные алгоритмы. Задание 23.}
\label{sec:task1}

\begin{enumerate}
	\item\largeТекущее время (часы, минуты, секунды) задано тремя переменными: \textit{h, m, s}. Округлить его до целых значений минут и часов. Например, 14 ч 21 мин 45 с преобразуется в 14 ч 22 мин или 14 ч, а 9 ч 59 мин 23 с — соответственно в 9 ч 59 мин или 10 ч.
	\begin{item}
		Словесное описание \textit{алгоритма}:

		\begin{enumerate}
			\item\largeЗнаем, что в одном часе 60 минут, а в одной минуте 60 секунд. Будем придерживаться принципа округления - если количество секунд $<50\%$, тогда округляем количество минут в меньшую сторону, иначе - в большую. С минутами в часах будем действовать по тому же принципу.
			\item\largeИсходя из вышесказанного, если количество секунд $<30$, тогда округляем минуты в меньшую сторону, иначе - в большую. С минутами в часах действуем по тому же принципу.
			\item\largeДля решения сначала запишем результаты округлений в соответствующие переменные, а потом объединим их в общую.
		\end{enumerate}
	\end{item}
	\begin{item}
		Спецификация функции \textit{TimeRounding}:
		\begin{enumerate}
			\itemЗаголовок: \colorbox{pybg}{\textit{def} TimeRounding(hours: int, minutes: int, seconds: int) -> str:}
			\itemНазначение: используется для нахождения целого числа минут и часов или только часов по введенному времени.
		\end{enumerate}
		\newpage
		Блок-схема:
		\begin{center}
			\begin{tikzpicture}[node distance=1.2cm]
				\node (start) [terminator, minimum width=15.6cm]
				{\textit{def} TimeRounding(hours: int, minutes: int, seconds: int)};

				\node (compute_only_hours) [process, below of=start, yshift=-0.5cm, minimum width=15.6cm]
				{\begin{tabular}{c}Приведем переменную \textbf{hours} к строковому виду с округлением по минутам \\ из переменной \textbf{minutes}, с сохранением результата в переменную \textbf{only\_hours}\end{tabular}};

				\node (compute_hours_with_minutes) [process, below of=compute_only_hours, yshift=-1cm, minimum width=15.6cm]
				{\begin{tabular}{c}Приведем переменные \textbf{hours} и \textbf{minutes} к строковому \\ виду с округлением по секундам из переменной \textbf{seconds}, \\ с сохранением результата в переменную \textbf{hours\_with\_minutes}\end{tabular}};

				\node (compute_result) [process, below of=compute_hours_with_minutes, yshift=-1cm, minimum width=15.6cm]
				{\begin{tabular}{c}Объединим переменные \textbf{only\_hours} и \textbf{hours\_with\_minutes} \\ и запишем результат в переменную \textbf{result}\end{tabular}};

				\node (return) [data, below of=compute_result, yshift=-0.5cm, minimum width=7cm]
				{Возврат result};

				\node (stop) [terminator, below of=return, yshift=-0.15cm, minimum width=15.6cm]
				{Конец};

				\path [connector] (start) -- (compute_only_hours);
				\path [connector] (compute_only_hours) -- (compute_hours_with_minutes);
				\path [connector] (compute_hours_with_minutes) -- (compute_result);
				\path [connector] (compute_result) -- (return);
				\path [connector] (return) -- (stop);
			\end{tikzpicture}
		\end{center}
	\end{item}
	\begin{item}
		Код \textit{алгоритма} на языке \textit{Python}:
		\begin{mycode}
def TimeRounding(hours: int, minutes: int, seconds: int) -> str:
    only_hours: str = f'{hours + 1 if minutes >= 30 else hours} ч'
    hours_with_minutes: str = \
        f'{hours} ч {minutes + 1 if seconds >= 30 else minutes} м'
    result: str = f'{hours_with_minutes} или {only_hours}'

    return result
		\end{mycode}
	\end{item}
	\begin{item}
		\hfill \textit{Таблица 1}

		\centering\textbf{Тестовые данные}

		\begin{table}[h]
			\begin{center}
				\begin{large}
					\begin{tabularx}{\textwidth}{>{\vspace{1pt}}X<{\vspace{4pt}}|>{\vspace{1pt}}X<{\vspace{4pt}}}
						\hline
						Входные данные & Выходные данные \\ \hline
						\makecell[l]{\begin{tabular}{l}14 \\ 21 \\ 45\end{tabular}} & \makecell[l]{14 ч 22 м или 14 ч} \\ \hline
						\makecell[l]{\begin{tabular}{l}9 \\ 59 \\ 23\end{tabular}} & \makecell[l]{9 ч 59 м или 10 ч} \\ \hline
					\end{tabularx}
				\end{large}
			\end{center}
		\end{table}
	\end{item}
\end{enumerate} \newpage

		\section{Тема 2. Разветвляющиеся алгоритмы. Задание 23.}
\label{sec:task2}

\begin{enumerate}
	\item\largeДля заданного 0 < \textit{n} $\leq$ 200, рассматриваемого как возраст человека, вывести фразу вида: «Мне 21 год», «Мне 32 года», «Мне 12 лет».
	\begin{item}
		Словесное описание \textit{алгоритма}:

		\begin{enumerate}
			\item\largeЕсли $n \% 10 = 0$, $n \% 10 \geq 5$ или $10 \leq n \leq 20$ - мы пишем «лет». Если \textit{n} оканчивается на 1, при этом $n \neq 11$, тогда мы пишем «год». В остальных случаях мы пишем «года».
			\item\largeИсходя из вышесказанного, запишем условия для получения правильной формы слова.
			\item\largeПолучив правильную форму слова, объединим ее с шаблоном предложения.
		\end{enumerate}
	\end{item}
	\begin{item}
		Спецификация функции \textit{ReturnAgeText}:
		\begin{enumerate}
			\itemЗаголовок: \colorbox{pybg}{\textit{def} ReturnAgeText(age: int) -> str}
			\itemНазначение: используется для нахождения правильной формы слова «лет», обозначающего возраст, в шаблон «Мне \textit{n} лет».
		\end{enumerate}
		\newpage
		Блок-схема:
		\begin{center}
			\begin{tikzpicture}[node distance=1.2cm]
				\node (start) [terminator]
				{\textit{def} ReturnAgeText(age: int)};

                \node (if)[decision, below of=start, node distance = 4cm]
                {\begin{tabular}{c}$0 < age \% 100 < 20$ \\ or $age \% 10 >= 5$ \\ or $age \% 10 = 0$\end{tabular}};

                \node (if_true)[process, below right = 1cm of if]
                {correctForm := «лет»};

				\node (second_if)[decision, below left = 2cm of if, yshift = 1cm, xshift = -0.5cm]
				{$age \% 10 = 1$};

                \node (second_if_true)[process, below right = 0.7cm of second_if]
                {correctForm := «год»};

				\node (second_if_false)[process, below left = 0.7cm of second_if]
				{correctForm := «года»};

				\node (result)[process, below of = if, node distance = 7cm]
				{\begin{tabular}{c}Объединим правильную форму слова с шаблоном, \\ результат запишем в перменную \textbf{result}\end{tabular}};

				\node (return) [data, below of=result, yshift=-0.4cm, minimum width=5.53cm]
				{Возврат result};

				\node (stop) [terminator, below of=return, minimum width=5.53cm]
				{Конец};

                \node at($(if) - (3.5, -0.4)$){\bf--};
				\node at($(if) - (-3.5, -0.4)$){\bf+};
                \node at($(second_if) - (2.4, -0.4)$){\bf--};
				\node at($(second_if) - (-2.4, -0.4)$){\bf+};

				\path [connector] (start) -- (if);
				\path [connector] (if.east) -- ++(1, 0) -| (if_true.north);
				\path [connector] (if.west) -- ++(-1, 0) -| (second_if.north);
				\path [connector] (second_if.east) -- ++(1.5, 0) -| (second_if_true.north);
				\path [connector] (second_if.west) -- ++(-1.8, 0) -| (second_if_false.north);

                \path [connector] (if_true) -- ++(0, -3) -| (result.north);
                \path [connector] (second_if_false) -- ++(0, -1) -- ++(3.5, 0) -- ++(0, -0.245) -| (result.north);
                \path [connector] (second_if_true) -- ++(0, -1) -- ++(-3.39, 0) -- ++(0, -0.245) -| (result.north);

                \path [connector] (result) -- (return);
                \path [connector] (return) -- (stop);
			\end{tikzpicture}
		\end{center}
	\end{item}
	\begin{item}
		Код \textit{алгоритма} на языке \textit{Python}:
		\begin{mycode}
def ReturnAgeText(age: int) -> str:
    if (10 < age % 100 < 20) or (age % 10 >= 5) or (age % 10 == 0):
        correctForm: str = 'лет'
    elif age % 10 == 1:
        correctForm: str = 'год'
    else:
        correctForm: str = 'года'
    answer: str = f'Мне {age} {correctForm}'

    return answer
		\end{mycode}
	\end{item}
	\newpage
	\begin{item}
		\hfill \textit{Таблица 2}

		\centering\textbf{Тестовые данные}

		\begin{table}[h]
			\begin{center}
				\begin{large}
					\begin{tabularx}{\textwidth}{>{\vspace{1pt}}X<{\vspace{4pt}}|>{\vspace{1pt}}X<{\vspace{4pt}}}
						\hline
						Входные данные & Выходные данные \\ \hline
						\makecell[l]{11} & \makecell[l]{Мне 11 лет} \\ \hline
						\makecell[l]{1} & \makecell[l]{Мне 1 год} \\ \hline
						\makecell[l]{21} & \makecell[l]{Мне 21 год} \\ \hline
						\makecell[l]{30} & \makecell[l]{Мне 30 лет} \\ \hline
						\makecell[l]{42} & \makecell[l]{Мне 42 года} \\ \hline
					\end{tabularx}
				\end{large}
			\end{center}
		\end{table}
	\end{item}
\end{enumerate} \newpage

		\section{Тема 3. Циклические и итерационные алгоритмы. Задание 23.}
\label{sec:task3}

\begin{enumerate}
	\item\largeЛеспромхоз ведёт заготовку деловой древесины. Первоначальный объём её на территории леспромхоза составлял \textit{p} кубометров. Ежегодный прирост составляет \textit{k}\%. Годовой план заготовки — \textit{t} кубометров. Через сколько лет в бывшем лесу будут расти одни опята?
	\begin{item}
		Словесное описание \textit{алгоритма}:

		\begin{enumerate}
			\item\largeИзвестно, что ежемесячно объем древесины \textit{p} кубометров увеличивается на \textit{k}\%, при этом уменьшался на \textit{t} кубометров ежегодно.
			\item\largeИсходя из условия, самый простой способ узнать срок жизни леса - простой перебор, который будет считать, сколько древесины остается спустя каждый год.
			\item\largeОрганизуем цикл, который будет отслеживать количество древесины спустя каждый год. Если количество ее кубометров будет = 0, значит срок жизни леса подошел к концу.
		\end{enumerate}
	\end{item}
	\begin{item}
		Спецификация функции \textit{CalculateAgeDeathForest}:
		\begin{enumerate}
			\itemЗаголовок: \colorbox{pybg}{\textit{def} CalculateAgeDeathForest(area: int, growth: int, plan: int) -> int}
			\itemНазначение: используется для нахождения срока жизни леса в годах.
		\end{enumerate}
		Блок-схема:
		\begin{center}
			\begin{tikzpicture}[node distance=1.2cm]
                \node (start) [terminator]
				{\textit{def} CalculateAgeDeathForest(area: int, growth: int, plan: int)};

				\node (init_countAge)[process, below of=start]
                {countAge := 0};

                \node (while)[decision, below of=init_countAge, yshift=-0.8cm]
                {area > 0};

                \node (while_true1)[process, below right = 0.5cm of while]
                {area := area * (1 + growth / 100) - plan};

                \node (while_true2)[process, below of=while_true1, minimum width=7.71cm]
                {countAge := countAge + 1};

                \node (return) [data, below of=while, yshift=-3cm]
				{Возврат countAge};

				\node (stop) [terminator, below of=return, minimum width=11.5cm]
				{Конец};

				\node at($(while) - (3, -0.4)$){\bf--};
				\node at($(while) - (-3, -0.4)$){\bf+};

                \path [connector] (start) -- (init_countAge);
                \path [connector] (init_countAge) -- (while);
                \path [connector] (while.east) -| (while_true1.north);
                \path [connector] (while_true1) -- (while_true2);
                \path [connector] (while.west) -- ++(-3.4, 0) -- ++(0, -3.4) -| (return.north);
                \path [connector] (while_true2) -- ++(0, -0.83) -| (return.north);
                \path [connector] (return) -- (stop);
			\end{tikzpicture}
		\end{center}
	\end{item}
	\newpage
	\begin{item}
		Код \textit{алгоритма} на языке \textit{Python}:
		\begin{mycode}
def CalculateAgeDeathForest(area: int, growth: int, plan: int) -> int:
    countAge: int = 0
    while area > 0:
        area = area * (1 + growth / 100) - plan
        countAge += 1

    return countAge
		\end{mycode}
	\end{item}
	\begin{item}
		\hfill \textit{Таблица 3}

		\centering\textbf{Тестовые данные}

		\begin{table}[h]
			\begin{center}
				\begin{large}
					\begin{tabularx}{\textwidth}{>{\vspace{1pt}}X<{\vspace{4pt}}|>{\vspace{1pt}}X<{\vspace{4pt}}}
						\hline
						Входные данные & Выходные данные \\ \hline
						\makecell[l]{\begin{tabular}{l}10 \\ 9 \\ 4\end{tabular}} & \makecell[l]{3} \\ \hline
						\makecell[l]{\begin{tabular}{l}10 \\ 10 \\ 4\end{tabular}} & \makecell[l]{4} \\ \hline
						\makecell[l]{\begin{tabular}{l}1 \\ 100 \\ 2\end{tabular}} & \makecell[l]{1} \\ \hline
						\makecell[l]{\begin{tabular}{l}0 \\ 1000 \\ 1\end{tabular}} & \makecell[l]{0} \\ \hline
					\end{tabularx}
				\end{large}
			\end{center}
		\end{table}
	\end{item}
\end{enumerate} \newpage

		\section{Тема 4. Простейшие операции над массивами. Задание 23.}
\label{sec:task4}

\begin{enumerate}
	\item\largeКаждый из элементов $x_i$ массива \textit{X(n)} заменить средним значением первых \textit{i} элементов этого массива.
	\begin{item}
		Словесное описание \textit{алгоритма}:

		\begin{enumerate}
			\item\largeДля того, чтобы заменить каждый элемент массива средним значением первых \textit{i} элементов массива, необходимо найти это среднее значение.
			\item\largeПросуммируем первые \textit{i} элементов массива и разделим сумму на \textit{i}, чтобы получить их среднее значение.
			\item\largeИмея среднее значение первых \textit{i} элементов массива, заменим каждый его член на это значение.
		\end{enumerate}
	\end{item}
	\begin{item}
		Спецификация функции \textit{ReplacingArrayElements}:
		\begin{enumerate}
			\itemЗаголовок: \colorbox{pybg}{\begin{tabular}{l}\textit{def} ReplacingArrayElements(lengthArray: int, array: list, \\ countFirstNumbers: int) -> None:\end{tabular}}
			\itemНазначение: используется для замены элементов массива на среднее значение его первых \textit{i} элементов.
		\end{enumerate}
		Блок-схема:
		\begin{center}
			\begin{tikzpicture}[node distance=1.2cm]
                \node (start) [terminator]
				{\textit{def} ReplacingArrayElements(lengthArray: int, array: list, countFirstNumbers: int)};

                \node (init_firstNumsSum)[process, below of=start, minimum width=11cm]
                {firstNumsSum := 0};

				\node (for_1)[cycle, below of=init_firstNumsSum, minimum width=11cm]
				{$ind1 := 0 \dots countFirstNumbers$};

				\node (sum_nums)[process, below of=for_1, minimum width=11cm]
                {firstNumsSum := firstNumsSum + array(ind1)};

                \node (init_firstNumsAverage)[process, below of=sum_nums, minimum width=11cm, yshift=-1cm]
                {firstNumsAverage := firstNumsSum / countFirstNumbers};

                \node (for_2)[cycle, below of=init_firstNumsAverage, minimum width=11cm]
				{$ind2 := 0 \dots lengthArray$};

				\node (change)[process, below of=for_2, minimum width=11cm]
                {array(ind2) := firstNumsAverage};

                \node (stop) [terminator, below of=change, minimum width=15.25cm, yshift=-1cm]
				{Конец};

				\node at($(for_1) - (-6.3, -0.6)$){\begin{tabular}{l}Конец \\ цикла\end{tabular}};
				\node at($(for_2) - (-6.3, -0.6)$){\begin{tabular}{l}Конец \\ цикла\end{tabular}};

                \path [connector] (start) -- (init_firstNumsSum);
                \path [connector] (init_firstNumsSum) -- (for_1);
                \path [connector] (for_1) -- (sum_nums);
                \path [connector] (sum_nums.south) -- ++(0, -0.5) -- ++(-7, 0) |- (for_1.west);
                \path [connector] (for_1.east) -- ++(1.5, 0) -- ++(0, -2.4) -| (init_firstNumsAverage.north);
                \path [connector] (init_firstNumsAverage) -- (for_2);
                \path [connector] (for_2) -- (change);
                \path [connector] (change.south) -- ++(0, -0.5) -- ++(-7, 0) |- (for_2.west);
                \path [connector] (for_2.east) -- ++(1.5, 0) -- ++(0, -2.4) -| (stop.north);
			\end{tikzpicture}
		\end{center}
	\end{item}
	\newpage
	\begin{item}
		Код \textit{алгоритма} на языке \textit{Python}:
		\begin{mycode}
def ReplacingArrayElements(lengthArray: int, array: list,
                           countFirstNumbers: int) -> None:
    if (countFirstNumbers > lengthArray) | (countFirstNumbers == 0):
        return

    firstNumsSum: int = 0
    for ind in range(0, countFirstNumbers):
        firstNumsSum += array[ind]

    firstNumsAverage: float = firstNumsSum / countFirstNumbers

    for ind in range(0, lengthArray):
        array[ind] = firstNumsAverage
		\end{mycode}
	\end{item}
	\begin{item}
		\hfill \textit{Таблица 4}

		\centering\textbf{Тестовые данные}

		\begin{table}[h]
			\begin{center}
				\begin{large}
					\begin{tabularx}{\textwidth}{>{\vspace{1pt}}X<{\vspace{4pt}}|>{\vspace{1pt}}X<{\vspace{4pt}}}
						\hline
						Входные данные & Выходные данные \\ \hline
						\makecell[l]{\begin{tabular}{l}5 \\ 1 2 3 4 5 \\ 6\end{tabular}} & \makecell[l]{1 2 3 4 5} \\ \hline
						\makecell[l]{\begin{tabular}{l}5 \\ 1 2 3 4 5 \\ 3\end{tabular}} & \makecell[l]{2 2 2 2 2} \\ \hline
						\makecell[l]{\begin{tabular}{l}5 \\ 1 2 3 4 5 \\ 0\end{tabular}} & \makecell[l]{1 2 3 4 5} \\ \hline
					\end{tabularx}
				\end{large}
			\end{center}
		\end{table}
	\end{item}
\end{enumerate} \newpage

		\section{Тема 5. Векторы и матрицы. Задание 23.}
\label{sec:task5}

\begin{enumerate}
	\item\largeМногочлены $P_n(x)$ и $Q_m(x)$ заданы своими коэффициентами. Определить коэффициенты их композиции — многочлена $P_n(Q_m(x))$.
	\begin{item}
		Словесное описание \textit{алгоритма}:

		\begin{enumerate}
			\item\largeДля начала, чтобы получить коэффициенты композиции $P_n(Q_m(x))$, нам необходимо подставить на место \textit{x} в многочлен $P_n(x)$ многочлен $Q_m(x)$.
			\item\largeПосле замены переменной \textit{x} нам необходимо лишь выполнить вычисления для упрощения многочлена.
			\item\largeВышесказанного достаточно, чтобы получить композицию многочленов и выразить ее коэффициенты. Существуют два частных случая:
			    \begin{enumerate}[label=\arabic*.]
			        \item\largeЕсли многочлен $P_n(x)$ нулевой - тогда и композиция будет нулевой.
			        \item\largeЕсли многочлен $Q_m(x)$ нулевой - тогда у композиции будет один коэффициент, равный свободному члену многочлена $P_n(x)$.
			    \end{enumerate}
		\end{enumerate}
	\end{item}
	\begin{item}
		Спецификация функции \textit{ComposePolynomss}:
		\begin{enumerate}
			\itemЗаголовок: \colorbox{pybg}{\begin{tabular}{l}\textit{def} ComposePolynoms(coefsP: list | None,\\ coefsQ: list | None) -> list | None\end{tabular}}
			\itemНазначение: используется для нахождения коэффициентов композиции двух многочленов.
		\end{enumerate}
		\newpage
		Блок-схема:
		\begin{center}
			\begin{tikzpicture}[node distance=1.2cm]
                \node (start) [terminator, minimum width=14cm]
				{\textit{def} ComposePolynoms(coefsP: list | None, coefsQ: list | None)};

				\node (check_P) [process, below of=start, yshift=-0.5cm, minimum width=14cm]
				{\begin{tabular}{c}Если многочлен $P_n(x)$ нулевой, тогда запишем в переменную \textbf{result} то, \\ что коэффициенты композиции нулевые (отметим это None)\end{tabular}};

                \node (check_Q) [process, below of=check_P, yshift=-1.2cm, minimum width=14cm]
				{\begin{tabular}{c}Если многочлен $Q_m(x)$ нулевой, а $P_n(x)$ не нулевой, тогда \\ запишем в переменную \textbf{result} то, что коэффициентом \\ композиции будет свободный член многочлена $P_n(x)$\end{tabular}};

                \node (other) [process, below of=check_Q, yshift=-1.8cm, minimum width=14cm]
				{\begin{tabular}{c}Если оба многочлена не нулевые, тогда воспользуемся библиотекой \\ numpy для нахождения коэффициентов композиции многочленов \\ $P_n(x)$ и $Q_m(x)$, после чего запишем полученный результат \\ в переменную \textbf{result}\end{tabular}};

                \node (return) [data, below of=other, yshift=-1.9cm, minimum width=14cm]
				{Возврат result};

                \node (stop) [terminator, below of=return, yshift=-1cm, minimum width=14cm]
				{Конец};

                \path [connector] (start) -- (check_P);
                \path [connector] (check_P) -- (check_Q);
                \path [connector] (check_Q) -- (other);
                \path [connector] (other) -- (return);
                \path [connector] (return) -- (stop);
			\end{tikzpicture}
		\end{center}
	\end{item}
	\begin{item}
		Код \textit{алгоритма} на языке \textit{Python}:
		\begin{mycode}
def ComposePolynoms(coefsP: list | None, coefsQ: list | None) -> list | None:
    if coefsP is None:
        result: None = None
    elif coefsQ is None:
        result: list = [coefsP[-1]]
    else:
        PolynomP = np.poly1d(coefsP)
        PolynomQ = np.poly1d(coefsQ)
        composedPolynom = PolynomP(PolynomQ)
        result: list = composedPolynom.coefficients.tolist()

    return result
		\end{mycode}
	\end{item}
	\newpage
	\begin{item}
		\hfill \textit{Таблица 5}

		\centering\textbf{Тестовые данные}

		\begin{table}[h]
			\begin{center}
				\begin{large}
					\begin{tabularx}{\textwidth}{>{\vspace{1pt}}X<{\vspace{4pt}}|>{\vspace{1pt}}X<{\vspace{4pt}}}
						\hline
						Входные данные & Выходные данные \\ \hline
						\makecell[l]{\begin{tabular}{l}2 0 -5 \\ 1 -1 2\end{tabular}} & \makecell[l]{2 -4 10 -8 3} \\ \hline
						\makecell[l]{\begin{tabular}{l}1 2 \\ 2 -5\end{tabular}} & \makecell[l]{2 -3} \\ \hline
						\makecell[l]{\begin{tabular}{l}1 \\ 2 0 11\end{tabular}} & \makecell[l]{1} \\ \hline
						\makecell[l]{\begin{tabular}{l}None \\ 2 0 11\end{tabular}} & \makecell[l]{None} \\ \hline
						\makecell[l]{\begin{tabular}{l}2 0 11 \\ None \end{tabular}} & \makecell[l]{11} \\ \hline
					\end{tabularx}
				\end{large}
			\end{center}
		\end{table}
	\end{item}
\end{enumerate} \newpage

		\section{Тема 6. Линейный поиск. Задание 23.}
\label{sec:task6}

\begin{enumerate}
	\item\largeВ массиве \textit{P(n)} найти самую длинную последовательность, которая является арифметической или геометрической прогрессией.
	\begin{item}
		Словесное описание \textit{алгоритма}:

		\begin{enumerate}
			\item\largeТак как любые два числа образуют прогрессию, получаем коэффициент для поиска прогрессии от них.
			\item\largeИдем по элементам до тех пор, пока соблюдаются условия прогрессии.
			\item\largeКогда прогрессия прерывается, берем ее последние два числа, получаем новый коэффициент и ищем новую прогрессию.
		\end{enumerate}
	\end{item}
	\begin{item}
		Спецификация функции \textit{SearchLargestArithOrGeomProgression}:
		\begin{enumerate}
			\itemЗаголовок: \colorbox{pybg}{\begin{tabular}{l}\textit{def} SearchLargestArithOrGeomProgression(lengthArray: int, \\ array: list) -> int\end{tabular}}
			\itemНазначение: используется для нахождения наибольшей длины алгоритмической или геометрической прогрессии в массиве \textit{P(n)}.
		\end{enumerate}
		\newpage
		Блок-схемы:
		\begin{center}
			\begin{tikzpicture}[node distance=1.2cm]
                \node (start) [terminator]
				{\textit{def} SearchLargestArithOrGeomProgression(lengthArray: int, array: list)};

				\node (if)[decision, below of=start, yshift=-2.2cm]
                {\begin{tabular}{c}lengthArray \\ $\leq$ \\ 2\end{tabular}};

                \node (if_true)[process, below right = 0.7cm of if]
                {largestProgression := lengthArray};

				\node (if_false1)[process, below left = 0.7cm of if, minimum width=6.4cm]
				{\begin{tabular}{c}largestArith := \\ SearchLargestProgression(\\ array, lengthArray, True)\end{tabular}};

				\node (if_false2)[process, below of=if_false1, yshift=-1.5cm, minimum width=6.4cm]
				{\begin{tabular}{c}largestGeom := \\ SearchLargestProgression(\\ array, lengthArray, False)\end{tabular}};

				\node (if_false3)[process, below of=if_false2, yshift=-2.4cm]
				{\begin{tabular}{c}Используем функцию max() \\ для определения наибольшего \\ числа между largestArith и \\ largestGeom. Результат \\ записываем в переменную \\ \textbf{largestProgression}\end{tabular}};

                \node (return) [data, below of=if, yshift=-12cm]
				{Возврат largestProgression};

				\node (stop) [terminator, below of=return, minimum width=13.42cm]
				{Конец};

                \node at($(if) - (3.6, -0.4)$){\bf--};
				\node at($(if) - (-3.6, -0.4)$){\bf+};

                \path [connector] (start) -- (if);
                \path [connector] (if.east) -| (if_true.north);
                \path [connector] (if.west) -| (if_false1.north);
                \path [connector] (if_false1) -- (if_false2);
                \path [connector] (if_false2) -- (if_false3);
                \path [connector] (if_false3.south) -- ++(0, -1) -| (return);
                \path [connector] (if_true.south) -- ++(0, -9.518) -| (return);
                \path [connector] (return) -- (stop);
			\end{tikzpicture}

			\newpage

			\begin{tikzpicture}[node distance=1.2cm]
			    \node (start) [terminator, minimum width=15.1cm]
				{\textit{def} SearchLargestProgression(array: list, length: int, isArith: bool)};

				\node (actionSymbol)[process, below of=start, yshift=-0.3cm, minimum width=15.1cm]
				{\begin{tabular}{c}Если \textbf{isArith}, тогда результатом будет «-», иначе - «/». \\ Результат запишем в переменную \textbf{actionSymbol}\end{tabular}};

                \node (largest)[process, below of=actionSymbol, yshift=-0.3cm, minimum width=15.1cm]
				{largest := 0};

                \node (currentLength)[process, below of=largest, minimum width=15.1cm]
				{currentLength := 2};

				\node (currentCoef)[process, below of=currentLength, yshift=-0.3cm, minimum width=15.1cm]
				{\begin{tabular}{c}Воспользуемся eval() и получим коэффициент между вторым и \\ первым элементами. Результат запишем в переменную \textbf{currentCoef}\end{tabular}};

                \node (for)[cycle, below of=currentCoef, yshift=-0.3cm]
				{$ind := 2 \dots length$};

				\node (actuallyCoef)[process, below of=for, yshift=-0.3cm]
				{\begin{tabular}{c}Воспользуемся eval() и получим коэффициент между a(ind) и \\ a(ind - 1). Результат запишем в переменную \textbf{actuallyCoef}\end{tabular}};

                \node (if)[decision, below of=actuallyCoef, yshift=-2.4cm]
                {\begin{tabular}{c}actuallyCoef \\ = \\ currentCoef\end{tabular}};

                \node (if_true)[process, below right = 0.1cm of if]
                {currentLength := currentLength + 1};

                \node (if_false1)[process, below left = 0.1cm of if]
				{\begin{tabular}{c}Воспользуемся max() для \\ нахождения большего \\ числа между largest и \\ currentLength. Результат \\ запишем в largest\end{tabular}};

				\node (if_false2)[process, below of=if_false1, yshift=-1.2cm, minimum width=5.55cm]
				{currentLength := 2};

				\node (if_false3)[process, below of=if_false2, minimum width=5.55cm]
				{currentCoef := actuallyCoef};

				\node (last_check)[process, below of=if, yshift=-7.5cm]
				{Воспользуемся max() для largest и currentLength. Результат запишем в largest};

				\node (return) [data, below of=last_check, yshift=0.1cm]
				{Возврат largest};

				\node at($(if) - (3.25, -0.4)$){\bf--};
				\node at($(if) - (-3.6, -0.4)$){\bf+};
				\node at($(for) - (-5.25, -0.3)$){Конец цикла};

			    \path [connector] (start) -- (actionSymbol);
			    \path [connector] (actionSymbol) -- (largest);
			    \path [connector] (largest) -- (currentLength);
			    \path [connector] (currentLength) -- (currentCoef);
			    \path [connector] (currentCoef) -- (for);
			    \path [connector] (for) -- (actuallyCoef);
			    \path [connector] (actuallyCoef) -- (if);
			    \path [connector] (if) -| (if_true);
			    \path [connector] (if) -| (if_false1);
			    \path [connector] (if_false1) -- (if_false2);
			    \path [connector] (if_false2) -- (if_false3);
			    \path [connector] (if_false3.south) -- ++(0, -0.3) -- ++(4.15, 0) -- ++(0, -0.5) -- ++(-7.1, 0) |- (for);
			    \path [connector] (if_true.south) -- ++(0, -5.162) -- ++(-4.78, 0) -- ++(0, -0.5) -- ++(-7.1, 0) |- (for);
			    \path [connector] (for.east) -- ++(6.3, 0) -- ++ (0, -13.1) -| (last_check.north);
			    \path [connector] (last_check) -- (return);
			\end{tikzpicture}
		\end{center}
	\end{item}
	\newpage
	\begin{item}
		Код \textit{алгоритма} на языке \textit{Python}:
		\begin{mycode}
def SearchLargestArithOrGeomProgression(lengthArray: int, array: list) -> int:
    if lengthArray <= 2:
        largestProgression: int = lengthArray
    else:
        largestArith: int = SearchLargestProgression(
            array, lengthArray, True)
        largestGeom: int = SearchLargestProgression(
            array, lengthArray, False)
        largestProgression: int = max(largestArith, largestGeom)

    return largestProgression


def SearchLargestProgression(array: list, length: int, isArith: bool) -> int:
    actionSymbol: str = '-' if isArith else '/'
    largest: int = 0
    currentLength: int = 2
    currentCoef: float = eval(f'{array[1]} {actionSymbol} {array[0]}')
    for ind in range(2, length):
        actuallyCoef = eval(f'{array[ind]} {actionSymbol} {array[ind - 1]}')

        if actuallyCoef == currentCoef:
            currentLength += 1
        else:
            largest = max(largest, currentLength)
            currentLength = 2
            currentCoef = actuallyCoef

    largest = max(largest, currentLength)

    return largest
		\end{mycode}
	\end{item}
	\newpage
	\begin{item}
		\hfill \textit{Таблица 6}

		\centering\textbf{Тестовые данные}

		\begin{table}[h]
			\begin{center}
				\begin{large}
					\begin{tabularx}{\textwidth}{>{\vspace{1pt}}X<{\vspace{4pt}}|>{\vspace{1pt}}X<{\vspace{4pt}}}
						\hline
						Входные данные & Выходные данные \\ \hline
						\makecell[l]{\begin{tabular}{l}5 \\ 1 3 5 7 9\end{tabular}} & \makecell[l]{5} \\ \hline
						\makecell[l]{\begin{tabular}{l}5 \\ 2 1 3 5 9\end{tabular}} & \makecell[l]{3} \\ \hline
						\makecell[l]{\begin{tabular}{l}7 \\ 2 4 1 3 9 27 7\end{tabular}} & \makecell[l]{4} \\ \hline
						\makecell[l]{\begin{tabular}{l}4 \\ 3 4 11 2\end{tabular}} & \makecell[l]{2} \\ \hline
						\makecell[l]{\begin{tabular}{l}5 \\ 1 2 4 8 16\end{tabular}} & \makecell[l]{5} \\ \hline
						\makecell[l]{\begin{tabular}{l}2 \\ 1 12\end{tabular}} & \makecell[l]{2} \\ \hline
						\makecell[l]{\begin{tabular}{l}6 \\ 8 9 10 12 14 16\end{tabular}} & \makecell[l]{4} \\ \hline
					\end{tabularx}
				\end{large}
			\end{center}
		\end{table}
	\end{item}
\end{enumerate} \newpage

        \section{Тема 7. Разветвляющиеся алгоритмы. Задание 23.}
\label{sec:task7}

\begin{enumerate}
	\item\largeНайти все натуральные числа, не превосходящие заданного \textit{n}, десятичная запись которых есть строго возрастающая или строго убывающая последовательность цифр.
	\begin{item}
		Словесное описание \textit{алгоритма}:

		\begin{enumerate}
			\item\largeИдея решения заключается в том, чтобы рекурсивно составлять числа, добавляя к существующему цифры, и проверять их на соответствие условиям задачи.
			\item\largeДля реализации озвученной выше идеи создадим цикл для старта составления чисел с определенной цифры.
			\item\largeОпределенную цифру подаем в рекурсивную функцию, которая проверяет, не превышает ли оно заданного и является ли его десятичная запись строго возрастающей или строго убывающей последовательностью цифр.
			\item\largeСоответственно, если число будет подходить - тогда добавляем его к общей сумме таких чисел и продолжаем его составление дальше, в противном случае - заканчиваем эту рекурсивную ветвь.
			\item\largeВ конце нужно будет лишь просуммировать количество подходящих чисел от каждой начальной цифры и вернуть результат пользователю.
		\end{enumerate}
	\end{item}
	\begin{item}
		Спецификация функции \textit{SearchNaturalNumsWithSortedDigits}:
		\begin{enumerate}
			\itemЗаголовок: \colorbox{pybg}{\textit{def} SearchNaturalNumsWithSortedDigits(limit: int) -> int}
			\itemНазначение: используется для нахождения всех натуральных чисел, не превосходящих \textit{n}, десятичная запись которых - строго возрастающая или строго убывающая последовательность цифр.
		\end{enumerate}

		\newpage

		Блок-схемы:
		\begin{center}
			\begin{tikzpicture}[node distance=1.2cm]
                \node (start) [terminator]
				{\textit{def} SearchNaturalNumsWithSortedDigits(limit: int)};

				\node (init_countRightNums)[process, below of=start, minimum width=9.76cm]
				{countRightNums := 0};

				\node (for)[cycle, below of=init_countRightNums]
				{$startNum := 1 \dots 10$};

				\node (sum_result)[process, below of=for, yshift=-0.6cm]
				{\begin{tabular}{c}Вызовем \textbf{FindRightNums(str(startNum), limit, None)}, \\ передав \textbf{startNum} как строку, и добавим \\ возвращенный результат к \textbf{countRightNums}\end{tabular}};

                \node (return) [data, below of=for, yshift=-3.4cm]
				{Возврат countRightNums};

				\node (stop) [terminator, below of=return, minimum width=9.76cm]
				{Конец};

                \node at($(for) - (-4.5, -0.3)$){Конец цикла};

				\path [connector] (start) -- (init_countRightNums);
				\path [connector] (init_countRightNums) -- (for);
				\path [connector] (for) -- (sum_result);
				\path [connector] (sum_result.south) -- ++ (0, -0.6) -- ++ (-6.5, 0) |- (for.west);
				\path [connector] (for.east) -- ++ (4.2, 0) -- ++ (0, -3.7) -| (return.north);
				\path [connector] (return) -- (stop);
			\end{tikzpicture}

            \newpage

			\begin{tikzpicture}[node distance=1.2cm]
			\node (start) [terminator]
			{\textit{def} FindRightNums(num: str, limit: int, isIncreasing: bool | None)};

			\node (anyCheck)[process, below of=start, yshift=-0.7cm, minimum width=12.32cm]
			{\begin{tabular}{c}Проверяем число \textbf{num} на соответствие условиям, если хотя \\ бы одно из условий не проходит, значит True, иначе - False. \\ Результат записываем в переменную \textbf{anyCheck}\end{tabular}};

			\node (if)[decision, below of=anyCheck, yshift=-1.7cm]
            {anyCheck};

            \node (if_true)[process, below right = 1.5cm of if]
			{count := 0};

			\node (if_false1)[process, below left = 1.5cm of if]
			{count := 1};

			\node (if_false2)[process, below of=if_false1, yshift=-2cm]
			{\begin{tabular}{c}Если еще не известно, \textbf{isIncreasing} идет \\ на убывание или на возрастание, тогда \\ вызываем рекурсии на сборку и на \\ убывание, и на возрастание, иначе - на \\ то, что известно. Результаты вызовов \\ добавляем к переменной \textbf{count}\end{tabular}};

            \node (return) [data, below of=if, yshift=-8.5cm, minimum width=12.32cm]
			{Возврат count};

			\node at($(if) - (-2.1, -0.3)$){True};
			\node at($(if) - (2, -0.3)$){False};

            \path [connector] (start) -- (anyCheck);
            \path [connector] (anyCheck) -- (if);
            \path [connector] (if) -| (if_true);
            \path [connector] (if) -| (if_false1);
            \path [connector] (if_false1) -- (if_false2);
            \path [connector] (if_false2.south) -- ++ (0, -0.8) -| (return.north);
            \path [connector] (if_true.south) -- ++ (0, -5.57) -| (return.north);
			\end{tikzpicture}
		\end{center}
	\end{item}
	\newpage
	\begin{item}
		Код \textit{алгоритма} на языке \textit{Python}:
		\begin{mycode}
def SearchNaturalNumsWithSortedDigits(limit: int) -> int:
    countRightNums: int = 0
    for startNum in range(1, 10):
        countRightNums += FindRightNums(str(startNum), limit, None)

    return countRightNums


def FindRightNums(num: str, limit: int, isIncreasing: bool | None) -> int:
    numGreaterLimit: bool = (int(num) > limit)
    numIsStrictlyIncreasing: bool = (
        len(num) > 1 and int(num[-1]) <= int(num[-2])) if isIncreasing \
        else len(num) > 1 and int(num[-1]) >= int(num[-2])
    numIsBig: bool = len(num) > 10
    anyCheck: bool = numGreaterLimit or numIsStrictlyIncreasing or numIsBig

    if anyCheck:
        count: int = 0

    else:
        count: int = 1
        if isIncreasing or isIncreasing is None:
            for nextDigit in range(int(num[-1]), 10):
                count += FindRightNums(num + str(nextDigit), limit, True)
        if not isIncreasing or isIncreasing is None:
            for nextDigit in range(0, int(num[-1])):
                count += FindRightNums(num + str(nextDigit), limit, False)

    return count
		\end{mycode}
	\end{item}
	\newpage
	\begin{item}
		\hfill \textit{Таблица 7}

		\centering\textbf{Тестовые данные}

		\begin{table}[h]
			\begin{center}
				\begin{large}
					\begin{tabularx}{\textwidth}{>{\vspace{1pt}}X<{\vspace{4pt}}|>{\vspace{1pt}}X<{\vspace{4pt}}}
						\hline
						Входные данные & Выходные данные \\ \hline
						\makecell[l]{50} & \makecell[l]{46} \\ \hline
						\makecell[l]{100} & \makecell[l]{90} \\ \hline
						\makecell[l]{500} & \makecell[l]{174} \\ \hline
						\makecell[l]{1000} & \makecell[l]{294} \\ \hline
						\makecell[l]{10000000} & \makecell[l]{1458} \\ \hline
					\end{tabularx}
				\end{large}
			\end{center}
		\end{table}
	\end{item}
\end{enumerate} \newpage

        \section{Тема 8. Геометрия и теория множеств. Задание 23.}
\label{sec:task8}

\begin{enumerate}
	\item\largeМедианой множества точек на плоскости назовём прямую, которая делит множество на два подмножества одинаковой мощности. Найти горизонтальную и вертикальную медианы заданного множества, у которого никакие две точки не лежат на одной горизонтальной или вертикальной прямой.
	\begin{item}
		Словесное описание \textit{алгоритма}:

		\begin{enumerate}
			\item\largeДля решения задачи необходимо найти горизонтальную и вертикальную медианы, соответствующие условию. Для этого отсортируем точки сначала по x, затем - по y, взяв медианы из отсортированных точек.
			\item\largeПолученные медианы будут являться координатами проекции всех точек для построения медиан по условию.
		\end{enumerate}
	\end{item}
	\begin{item}
		Спецификация функции \textit{FindMedianXY}:
		\begin{enumerate}
			\itemЗаголовок: \colorbox{pybg}{\textit{def} FindMedianXY(countDots: int, arrayDots: list) -> list}
			\itemНазначение: используется для нахождения координат проекции всех точек прямой, являющихся медианами, которые делят вертикально и горизонтально множество точек на два равных по мощности подмножества.
		\end{enumerate}

		\newpage

		Блок-схемы:
		\begin{center}
			\begin{tikzpicture}[node distance=1.2cm]
                \node (start) [terminator]
			    {\textit{def} FindMedianXY(countDots: int, arrayDots: list)};

			    \node (init_halfLength)[process, below of=start, minimum width=9.54cm]
				{halfLength := countDots div 2};

				\node (init_arrayResult)[process, below of=init_halfLength, minimum width=9.54cm]
				{Создаем массив \textbf{arrayResult} для проекций};

				\node (if)[decision, below of=init_arrayResult, yshift=-2cm]
                {\begin{tabular}{c}countDots \\ > \\ 0\end{tabular}};

                \node (if_true1)[process, below right = 1.5cm of if]
			    {\begin{tabular}{c}Вызовем функцию \\ \textit{sortedArraysByIndexAndAddResult} для \\ нахождения медианы точек по \textit{x}, \\ это и будет координатой \\ проекции всех точек для  \\ вертикальной прямой. Результат \\ запишем в \textbf{arrayResult(0)}\end{tabular}};

                \node (if_true2)[process, below of=if_true1, yshift=-3.8cm]
			    {\begin{tabular}{c}Вызовем функцию \\ \textit{sortedArraysByIndexAndAddResult} для \\ нахождения медианы точек по \textit{y}, \\ это и будет координатой \\ проекции всех точек для \\ горизонтальной прямой. Результат \\ запишем в \textbf{arrayResult(1)}\end{tabular}};

                \node (return) [data, below of=if, yshift=-12.2cm]
			    {Возврат arrayResult};

                \node (stop) [terminator, below of=return, minimum width=9.54cm]
				{Конец};

				\node at($(if) - (-4.4, -0.4)$){\bf+};
				\node at($(if) - (3.8, -0.4)$){\bf--};

				\path [connector] (start) -- (init_halfLength);
				\path [connector] (init_halfLength) -- (init_arrayResult);
				\path [connector] (init_arrayResult) -- (if);
				\path [connector] (if) -| (if_true1);
				\path [connector] (if_true1) -- (if_true2);
				\path [connector] (if.west) -- ++(-3, 0) -- ++(0, -12.4) -| (return.north);
				\path [connector] (if_true2.south) -- ++(0, -0.575) -| (return.north);
				\path [connector] (return) -- (stop);
			\end{tikzpicture}

			\newpage
            \begin{tikzpicture}[node distance=1.2cm]
                \node (start) [terminator, minimum width=9.5cm]
			    {\begin{tabular}{l}\textit{def} sortedArraysByIndexAndAddResult(\\ arrayDots: list, arrayResult: list, \\ lengthArray: int, halfLength: int, keyInd: int)\end{tabular}};

                \node (sort)[process, below of=start, yshift=-1.5cm]
				{\begin{tabular}{c}Сортируем массив точек по ключу из \textbf{keyInd}. \\ Если $keyInd = 0$, тогда по \textit{х}, \\ а если $keyInd = 1$, тогда по \textit{y}\end{tabular}};

                \node (if)[decision, below of=sort, yshift=-2.6cm]
                {\begin{tabular}{c}lengthArray \\ mod 2 = 0\end{tabular}};

                \node (if_true)[process, below right = 0.5cm of if]
			    {\begin{tabular}{c}Получим среднее значение \\ двух чисел, стоящих по \\ середине вариационного \\ ряда, округлим до 2 \\ знаков после запятой \\ и запишем его в \\ переменную \textbf{result}\end{tabular}};

                \node (if_false)[process, below left = 0.5cm of if]
			    {\begin{tabular}{c}Возьмем число, стоящее \\ посередине вариационного \\ ряда и запишем его \\ в переменную \textbf{result}\end{tabular}};

                \node (append)[process, below of=if, yshift=-7cm, minimum width=9.5cm]
                {\begin{tabular}{c}Добавим переменную \textbf{result} \\ в массив \textbf{arrayResult}\end{tabular}};

                \node at($(if) - (-3.2, -0.4)$){\bf+};
				\node at($(if) - (3.2, -0.4)$){\bf--};

                \path [connector] (start) -- (sort);
                \path [connector] (sort) -- (if);
                \path [connector] (if) -| (if_true);
                \path [connector] (if) -| (if_false);
                \path [connector] (if_false.south) -- ++(0, -2.545) -| (append.north);
                \path [connector] (if_true.south) -- ++(0, -0.7) -| (append.north);
            \end{tikzpicture}
		\end{center}
	\end{item}
	\newpage
	\begin{item}
		Код \textit{алгоритма} на языке \textit{Python}:
		\begin{mycode}
def FindMedianXY(countDots: int, arrayDots: list) -> list:
    halfLength: int = countDots // 2
    arrayResult: list = []

    if countDots > 0:
        # вертикальная
        sortedArraysByIndexAndAddResult(arrayDots, arrayResult, countDots,
                                        halfLength, 0)

        # горизонтальная
        sortedArraysByIndexAndAddResult(arrayDots, arrayResult, countDots,
                                        halfLength, 1)

    return arrayResult


def sortedArraysByIndexAndAddResult(arrayDots: list, arrayResult: list,
                                    lengthArray: int, halfLength: int,
                                    keyInd: int) -> None:
    arrayDots.sort(key=lambda x: x[keyInd])
    if lengthArray % 2 == 0:
        result = round((arrayDots[halfLength][keyInd] +
                        arrayDots[halfLength - 1][keyInd]) / 2, 2)
    else:
        result = arrayDots[halfLength][keyInd]

    arrayResult.append(result)
		\end{mycode}
	\end{item}
	\newpage
	\begin{item}
		\hfill \textit{Таблица 8}

		\centering\textbf{Тестовые данные}

		\begin{table}[h]
			\begin{center}
				\begin{large}
					\begin{tabularx}{\textwidth}{>{\vspace{1pt}}X<{\vspace{4pt}}|>{\vspace{1pt}}X<{\vspace{4pt}}}
						\hline
						Входные данные & Выходные данные \\ \hline
						\makecell[l]{\begin{tabular}{l}4 \\ 3 4 \\ 1 2 \\ 7 8 \\ 5 6\end{tabular}} & \makecell[l]{4.00 5.00} \\ \hline
						\makecell[l]{\begin{tabular}{l}5 \\ 4 6 \\ 1 7 \\ 10 20 \\ 11 13 \\ 18 26\end{tabular}} & \makecell[l]{10 13} \\ \hline
						\makecell[l]{\begin{tabular}{l}1 \\ 10 20\end{tabular}} & \makecell[l]{10 20} \\ \hline
						\makecell[l]{\begin{tabular}{l}4 \\ 4.5 5.4 \\ 1.1 7.2 \\ 11.3 5.4 \\ 12.3 7.7\end{tabular}} & \makecell[l]{7.90 6.30} \\ \hline
					\end{tabularx}
				\end{large}
			\end{center}
		\end{table}
	\end{item}
\end{enumerate} \newpage

        \section{Тема 9. Линейная алгебра и сжатие информации. Задание 23.}
\label{sec:task9}

\begin{enumerate}
	\item\largeЗаданный неупакованный двоичный массив сжать, используя полубайтовое представление длин цепочек.
	\begin{item}
		Словесное описание \textit{алгоритма}:

		\begin{enumerate}
			\item\largeЗадача довольно непростая, так как подобные способы сжатия практически не используются. Разработаем для этого свой алгоритм.
			\item\largeБудем отталкиваться, что полбайта - это 4 бит. Максимальное число - 15.
			\item\largeСчитаем, что массив начинается с 1, то есть нет незначащих нулей.
			\item\largeБудем хранить в сжатом виде лишь количество подряд идущих бит, то есть цепочки.
			\item\largeПридерживаться будем следующих правил:
			    \begin{enumerate}[label=\arabic*.]
			    \item\largeЕсли длина цепочки меньше 15, тогда записываем фактическую длину цепочки.
			    \item\largeЕсли длина цепочки равна 15, тогда запишем в сжатый массив два числа - 15 и 0. 0 будет признаком того, что длина цепочки кратна 15. Дальше может идти лишь противоположный бит исходного массива.
			    \item\largeЕсли длина цепочки больше 15, тогда делаем переносы, записывая длину числами, не превышающими 15, до полного разложения исходной длины. Если после какого-то числа 15 идет не 0, значит это число также относится к цепочке. На примере: имеем 31 подряд идущий бит 0 в исходном массиве, а затем 3 подряд идущие 1. Запишем это так: [15, 15, 1, 3]. А если в исходном массиве будет 30 подряд идущих бит 0, а затем 3 подряд идущие 1, тогда запишем это так: [15, 15, 0, 3].
			    \end{enumerate}
		\end{enumerate}
	\end{item}
	\begin{item}
		Спецификация функции \textit{BinaryArrayCompression}:
		\begin{enumerate}
			\itemЗаголовок: \colorbox{pybg}{\textit{def} BinaryArrayCompression(array: list) -> list}
			\itemНазначение: используется для сжатия исходного неупакованного двоичного массива, применяя способ полубайтового представления длин цепочек.
		\end{enumerate}

		\newpage

		Блок-схемы:
		\begin{center}
			\begin{tikzpicture}[node distance=1.2cm]
                \node (start) [terminator, minimum width=8.772cm]
			    {\textit{def} BinaryArrayCompression(array: list)};

                \node (init_currentBit)[process, below of=start, minimum width=8.772cm]
				{currentBit := NULL};

				\node (init_currentBitCount)[process, below of=init_currentBit, minimum width=8.772cm]
				{currentBitCount := 0};

				\node (init_compressionArray)[process, below of=init_currentBitCount]
				{Создаем сжатый массив \textbf{compressionArray}};

				\node (for)[cycle, below of=init_compressionArray]
				{$bit := array(0) \dots array(n)$};

                \node (bit_action)[process, below of=for, yshift=-3.7cm]
				{\begin{tabular}{c}Если $currentBit = NULL$, тогда \\ запишем в нее \textit{bit} и добавим к \\ \textbf{currentBitCount} единицу. Если \\ $currentBit = bit$, тогда добавим \\ к \textbf{currentBitCount} единицу. Если \\ $currentBit \neq bit$, тогда вызовем \\ \textit{addChainLengthToCompressedArray}, \\ которая запишет длину \\ цепочки в сжатый массив, \\ сбросит \textbf{currentBitCount} до 1 \\ и запишет \textit{bit} в \textbf{currentBit}\end{tabular}};

                \node (last_signal)[process, below of=bit_action, yshift=-5.4cm, minimum width=8.772cm]
				{\begin{tabular}{c}Вызываем после цикла еще раз \\ \textit{addChainLengthToCompressedArray}, \\ чтобы занести в сжатый массив \\ длину последней цепочки, \\ которая не сбросилась\end{tabular}};

                \node (return) [data, below of=last_signal, yshift=-1.3cm, minimum width=8.772cm]
			    {Возврат compressionArray};

                \node (stop) [terminator, below of=return, minimum width=8.772cm]
				{Конец};

				\node at($(for) - (-4.2, -0.3)$){Конец цикла};

				\path [connector] (start) -- (init_currentBit);
				\path [connector] (init_currentBit) -- (init_currentBitCount);
				\path [connector] (init_currentBitCount) -- (init_compressionArray);
				\path [connector] (init_compressionArray) -- (for);
				\path [connector] (for) -- (bit_action);
				\path [connector] (bit_action.south) -- ++(0, -0.5) -- ++(-4.25, 0) |- (for.west);
				\path [connector] (for.east) -- ++(1.25, 0) -- ++(0, -9.2) -| (last_signal.north);
				\path [connector] (last_signal) -- (return);
				\path [connector] (return) -- (stop);
			\end{tikzpicture}

			\newpage

			\begin{tikzpicture}[node distance=1.2cm]
			    \node (start) [terminator]
			    {\begin{tabular}{l}\textit{def} addChainLengthToCompressedArray( \\ chainLength: int, compressedArray: list)\end{tabular}};

                \node (if1)[decision, below of=start, yshift=-2.2cm]
                {\begin{tabular}{c}chainLength \\ < 15\end{tabular}};

                \node (if1_true)[process, below right = 0.5cm of if1]
			    {\begin{tabular}{c}Запишем длину \\ цепочки в массив \\ \textbf{compressedArray}\end{tabular}};

                \node (if2)[decision, below left = 0.5cm of if1, xshift=-0.7cm]
                {\begin{tabular}{c}chainLength \\ = 15\end{tabular}};

                \node (if2_true)[process, below right = 0.4cm of if2]
			    {\begin{tabular}{c}Запишем в массив \\ \textbf{compressedArray} \\ 15 и 0\end{tabular}};

			    \node (if2_false)[process, below left = 0.4cm of if2]
			    {\begin{tabular}{c}Записываем в массив \\ \textbf{compressedArray} \\ длину цепочки \\ числами, не \\ превышающими 15, \\ раскладывая длину\end{tabular}};

                \node (stop) [terminator, below of=if1, yshift=-9cm, minimum width=8.3cm]
				{Конец};

				\node at($(if1) - (-2.9, -0.4)$){\bf+};
				\node at($(if1) - (2.7, -0.4)$){\bf--};

				\node at($(if2) - (-2.7, -0.4)$){\bf+};
				\node at($(if2) - (2.9, -0.4)$){\bf--};

                \path [connector] (start) -- (if1);
                \path [connector] (if1) -| (if1_true);
                \path [connector] (if1) -| (if2);
                \path [connector] (if2) -| (if2_true);
                \path [connector] (if2) -| (if2_false);
                \path [connector] (if1_true.south) -- ++(0, -5.85) -| (stop.north);
                \path [connector] (if2_true.south) -- ++(0, -2.8) -- ++(-3.58, 0) -- ++ (0, -0.552) -| (stop.north);
                \path [connector] (if2_false.south) -- ++(0, -0.955) -- ++(3.725, 0) -- ++ (0, -0.552) -| (stop.north);
			\end{tikzpicture}
		\end{center}
	\end{item}
	\newpage
	\begin{item}
		Код \textit{алгоритма} на языке \textit{Python}:
		\begin{mycode}
def BinaryArrayCompression(array: list) -> list:
    currentBit: int | None = None
    currentBitCount: int = 0
    compressionArray: list = []

    for bit in array:
        if currentBit is None:
            currentBit = bit
            currentBitCount += 1
        elif bit == currentBit:
            currentBitCount += 1
        else:
            addChainLengthToCompressedArray(currentBitCount, compressionArray)
            currentBitCount = 1
            currentBit = bit

    if currentBitCount > 0:
        addChainLengthToCompressedArray(currentBitCount, compressionArray)

    return compressionArray


def addChainLengthToCompressedArray(chainLength: int,
                                    compressedArray: list) -> None:
    if chainLength < 15:
        compressedArray.append(chainLength)
    elif chainLength == 15:
        compressedArray.append(chainLength)
        compressedArray.append(0)
    else:
        while chainLength > 0:
            if chainLength >= 15:
                numAdd = 15
            else:
                numAdd = chainLength

            chainLength -= numAdd
            compressedArray.append(numAdd)
		\end{mycode}
	\end{item}
	\newpage
	\begin{item}
		\hfill \textit{Таблица 9}

		\centering\textbf{Тестовые данные}

		\begin{table}[h]
			\begin{center}
				\begin{large}
					\begin{tabularx}{\textwidth}{>{\vspace{1pt}}X<{\vspace{4pt}}|>{\vspace{1pt}}X<{\vspace{4pt}}}
						\hline
						Входные данные & Выходные данные \\ \hline
						\makecell[l]{1 1 1 0 0 1} & \makecell[l]{3 2 1} \\ \hline
						\makecell[l]{1 0 1 0 1 0} & \makecell[l]{1 1 1 1 1 1} \\ \hline
						\makecell[l]{1 1 1 1 1 1 1 1 1 1 1 1 1 1 1 0 0} & \makecell[l]{15 0 2} \\ \hline
						\makecell[l]{1 1 1 1 1 1 1 1 1 1 1 1 1 1 1 1 1 0 1} & \makecell[l]{15 2 1 1} \\ \hline
					\end{tabularx}
				\end{large}
			\end{center}
		\end{table}
	\end{item}
\end{enumerate} \newpage

        \section{Тема 10. Алгоритмы обработки символьной информации. Задание 23.}
\label{sec:task10}

\begin{enumerate}
	\item\largeПо правилам пунктуации пробел может стоять после, а не перед каждым из следующих знаков: . , ; : ! ? ) ] \} …; перед, а не после знаков: ( [ \{. Заданный текст проверить на соблюдение этих правил и при необходимости исправить. Вместо пробела может быть перевод строки или знак табуляции.
	\begin{item}
		Словесное описание \textit{алгоритма}:

		\begin{enumerate}
			\item\largeУбираем лишние пробелы после знаков препинания: . , ; : ! ? ) ] \} …;
			\item\largeДобавляем недостающие пробелы после знаков препинания: . , ; : ! ? ) ] \} …;
			\item\largeУбираем пробелы после открывающих скобок: ( [ \{.
			\item\largeУбираем лишние пробелы перед открывающими скобками: ( [ \{.
		\end{enumerate}
	\end{item}
	\begin{item}
		Спецификация функции \textit{CorrectionText}:
		\begin{enumerate}
			\itemЗаголовок: \colorbox{pybg}{\textit{def} CorrectionText(text: str) -> str}
			\itemНазначение: используется для корректировки пробелов перед знаками препинания и после них, согласно правилам пунктуации.
		\end{enumerate}

		Блок-схема:
		\begin{center}
			\begin{tikzpicture}[node distance=1.2cm]
                \node (start) [terminator, minimum width=11.68cm]
			    {\textit{def} CorrectionText(text: str)};

                \node (main)[process, below of=start, yshift=-0.75cm]
				{\begin{tabular}{c}Воспользуемся функцией \textit{sub()} из модуля \textbf{re} и \\ скорректируем знаки препинания с помощью регулярных \\ выражений. Результат запишем в переменную \textbf{text}\end{tabular}};

                \node (return) [data, below of=main, yshift=-1.5cm, minimum width=11.68cm]
			    {Возврат text};

                \node (stop) [terminator, below of=return, yshift=-0.8cm, minimum width=11.68cm]
				{Конец};

				\path [connector] (start) -- (main);
				\path [connector] (main) -- (return);
				\path [connector] (return) -- (stop);
			\end{tikzpicture}
		\end{center}
	\end{item}
	\newpage
	\begin{item}
		Код \textit{алгоритма} на языке \textit{Python}:
		\begin{mycode}
def CorrectionText(text: str) -> str:
    # Убираем лишние пробелы после знаков препинания .,;:!?)]}
    text = re.sub(r'\s+([.,;:!?)}\]])', r'\1', text)

    # Добавляем недостающие пробелы после знаков препинания .,;:!?)]}
    text = re.sub(r'([.,;:!?)}\]])([^\s.,;:!?)}\]])', r'\1 \2', text)

    # Убираем пробелы после открывающих скобок
    text = re.sub(r'(\(\s+)', '(', text)
    text = re.sub(r'(\[\s+)', '[', text)
    text = re.sub(r'(\{\s+)', '{', text)

    # Убираем лишние пробелы перед открывающими скобками
    text = re.sub(r'\s+([\(\[\{])', r' \1', text)

    return text
		\end{mycode}
	\end{item}
	\begin{item}
		\hfill \textit{Таблица 10}

		\centering\textbf{Тестовые данные}

		\begin{table}[h]
			\begin{center}
				\begin{large}
					\begin{tabularx}{\textwidth}{>{\vspace{1pt}}X<{\vspace{4pt}}|>{\vspace{1pt}}X<{\vspace{4pt}}}
						\hline
						Входные данные & Выходные данные \\ \hline
						\makecell[l]{( Привет, Андрей ) , - говорил он} & \makecell[l]{(Привет, Андрей), - говорил он} \\ \hline
						\makecell[l]{(Да $\setminus$n? Ты в этом уверен ? )} & \makecell[l]{(Да? Ты в этом уверен?)} \\ \hline
						\makecell[l]{( ( Проверю на скобках ) )} & \makecell[l]{((Проверю на скобках))} \\ \hline
						\makecell[l]{Да , забавно (получается)} & \makecell[l]{Да, забавно (получается)} \\ \hline
						\makecell[l]{( Чек ) , посмотрим} & \makecell[l]{(Чек), посмотрим} \\ \hline
					\end{tabularx}
				\end{large}
			\end{center}
		\end{table}
	\end{item}
\end{enumerate} \newpage

        \section{Тема 11. Аналитическая геометрия. Задание 3.}
\label{sec:task11}

\begin{enumerate}
	\item\largeПостроить прямую $3x + 2y - 4 = 0$ в диапазоне $x \in [-1;3]$ с шагом $\Delta = 0,25$

    \begin{item}
		Нахождение уравнения прямой:
		\begin{enumerate}
			\item\largeВыразим \textit{y} из исходного уравнения. Перенесем \textit{y} в правую сторону и получим: $3x - 4 = -2y$
			\item\largeДомножим все уравнение на -1 и получим $-3x + 4 = 2y$
			\item\largeВыразим \textit{y} из полученного уравнения и получим $y = -1.5x + 2$
		\end{enumerate}
	\end{item}

	\begin{item}
		\largeПостроим таблицу данных (x; y). Значения аргумента заполним в диапазоне от -1 до 3 с шагом 0,25. В ячейку $B2$ вводим формулу вида = $-1,5 * A2 + 2$, затем автозаполнением получаем оставшиеся значения функции.

		\largeВ результате получена таблица, представленная на рисунке 1.
		\includeimage{assets/task_11_table.png}{0.22}{Таблица значений функции}

		\largeДля построения прямой выберем тип диаграммы -- \textit{График}, вид -- \textit{График с маркерами}. Полученный график представлен на рисунке 2.

		\includeimage{assets/task_11_graph.png}{0.8}{График прямой $y = -1.5x + 2$ в диапазоне $[-1;3]$}
	\end{item}
\end{enumerate} \newpage

        \section{Тема 12. Кривые второго порядка на плоскости. Задание 3.}
\label{sec:task12}

\begin{enumerate}
	\item\largeПостроить верхнюю часть эллипса $0,1 \leq x \leq 5,1$ с шагом $\Delta = 0,25$, заданного уравнением $\frac{x^2}{4} + y^2 = 1$

    \begin{item}
		Нахождение уравнения кривой:
		\begin{enumerate}
			\item\largeВыразим \textit{y} из исходного уравнения. Перенесем \textit{y} в правую сторону и получим: $1 - \frac{x^2}{4} = y^2$
			\item\largeВыразим \textit{y} из полученного уравнения и получим $y = \sqrt{1 - \frac{x^2}{4}}$
			\item\largeОбратим внимание, что значение выражения $1 - \frac{x^2}{4}$ должно быть $\geq 0$, а следовательно, допустимый диапазон аргумента $x \in [-2;2]$
		\end{enumerate}
	\end{item}

	\begin{item}
		\largeПостроим таблицу данных (x; y). Значения аргумента заполним в диапазоне от 0,1 до 5,1 с шагом 0,25. В ячейку $B2$ вводим формулу вида = $(1 - \frac{(A2) ^ 2}{4}) ^ \frac{1}{2}$, затем автозаполнением получаем оставшиеся значения функции. Значения функции в тех ячейках, при аргументе которых уравнение не имеет решений закрасим красным цветом. Поставим точку с аргументом $x = 2$ как крайнюю, имеющую решение.

		\largeВ результате получена таблица, представленная на рисунке 3.
		\includeimage{assets/task_12_table.png}{0.3}{Таблица значений функции}

        \newpage

		\largeДля построения прямой выберем тип диаграммы -- \textit{График}, вид -- \textit{График с маркерами}. Полученный график представлен на рисунке 4.

		\includeimage{assets/task_12_graph.png}{0.9}{График верхней части эллипса $\frac{x^2}{4} + y^2 = 1$ в диапазоне $[0,1;5,1]$}
	\end{item}
\end{enumerate} \newpage

        \section{Тема 13. Графическое решение систем уравнений. Задание 3.}
\label{sec:task13}

\begin{enumerate}
	\item\large
	Найдите точку равновесия в заданном диапазоне с заданным шагом.

		\[
			\begin{cases}
			    y = \frac{2}{x} & \text{в диапазоне } 0.1 \leq x \leq 4, \text{ с шагом } \Delta = 0.1 \\
			    y^2 = 2x
			\end{cases}
		\]
	\begin{item}
		\largeПостроим таблицу данных (x; y). Значения аргумента заполним в диапазоне от 0,1 до 4 с шагом 0,1. В ячейку $B3$ вводим формулу вида = $\frac{2}{x}$, для нахождения значений первой функции. Автозаполнением получаем оставшиеся значения функции. Вторая функция будет иметь значения с обоими знаками, так как \textit{y} в квадрате. Так же запишем значения аргумента в диапазоне от 0,1 до 4 с шагом 0,1, а в соседних колонках формулы $(2 * x) ^ \frac{1}{2}$ и $-((2 * x) ^ \frac{1}{2})$ соответственно, для нахождения значений с противоположными знаками.

		\largeВ результате получена таблица, представленная на рисунке 5.
		\includeimage{assets/task_13_table.png}{0.55}{Таблица значений функций}

        \newpage

		\largeДля построения графиков выберем тип диаграммы -- \textit{График}. Полученный график представлен на рисунке 6.

		\includeimage{assets/task_13_graph.png}{0.95}{Графики функций $y = \frac{2}{x}$ и $y^2 = 2x$ в диапазоне $[0,1;4]$}

		\largeПо графику мы приблизительно можем определить единственную точку пересечения функций -- (1,26; 1,58) с небольшой погрешностью. Для более точных значений необходимо либо уменьшить шаг, либо использовать другие средства поиска точки равновесия.
	\end{item}
\end{enumerate} \newpage

        \section{Тема 14. Плоскость в трехмерном пространстве. Задание 3.}
\label{sec:task14}

\begin{enumerate}
	\item\largeПостроить плоскость, проходящую через точки $M_1$(3,3,1), $M_2$(2,3,2), $M_3$(1,1,3), при $-1 \leq x \leq 4$ с шагом $\Delta = 0,5$ и $-1 \leq y \leq 3$ с шагом $\Delta = 1$.
	\item\largeПодставим значения под формулу $i * ((y_2 - y_1) * (z_3 - z_1) - (y_3 - y_1) * (z_2 - z_1)) - j * ((x_2 - x_1) * (z_3 - z_1) - (x_3 - x_1) * (z_2 - z_1)) + k * ((x_2 - x_1) * (y_3 - y_1) - (x_3 - x_1) * (y_2 - y_1))$, где номер переменной означает номер точки, из которой берется значение. Так получаем (2;0;2) -- вектор нормали к плоскости. Подставляем значения под общее уравнение плоскости $Ax + By + Cz + D = 0$ и получаем $2x + 2z + D = 0$. Возьмем значения любой из точек и получим, что $8 + D = 0$, следовательно $D = -8$. Выразим \textit{z} из уравнения $2x + 2z - 8 = 0$ и получим $z = 4 - x$.
	\begin{item}
		\largeПостроим таблицу данных (x; y; z). Значения \textit{x} заполним в диапазоне от -1 до 4 с шагом 0,5, а значения \textit{y} в диапазоне от -1 до 3 с шагом 1. Воспользуемся формулой $z = 4 - x$ и получим все значения \textit{z}.

		\largeВ результате получена таблица, представленная на рисунке 7.
		\includeimage{assets/task_14_table.png}{0.55}{Таблица значений функции $z(x, y)$}

        \newpage

		\largeДля построения плоскости выберем тип диаграммы -- \textit{Поверхность}, вид -- \textit{Проволочная поверхность}. Полученная плоскость представлена на рисунке 8.

		\includeimage{assets/task_14_graph.png}{0.9}{Диаграмма плоскости $2x + 2z - 8 = 0$}
	\end{item}
\end{enumerate} \newpage
	\end{onehalfspacing}
\end{document}